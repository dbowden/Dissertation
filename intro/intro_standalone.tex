\documentclass[12pt,]{book}
\usepackage[]{tgpagella}
\usepackage{amssymb,amsmath}
\usepackage{ifxetex,ifluatex}
\usepackage{fixltx2e} % provides \textsubscript
\ifnum 0\ifxetex 1\fi\ifluatex 1\fi=0 % if pdftex
  \usepackage[T1]{fontenc}
  \usepackage[utf8]{inputenc}
\else % if luatex or xelatex
  \ifxetex
    \usepackage{mathspec}
  \else
    \usepackage{fontspec}
  \fi
  \defaultfontfeatures{Ligatures=TeX,Scale=MatchLowercase}
\fi
% use upquote if available, for straight quotes in verbatim environments
\IfFileExists{upquote.sty}{\usepackage{upquote}}{}
% use microtype if available
\IfFileExists{microtype.sty}{%
\usepackage{microtype}
\UseMicrotypeSet[protrusion]{basicmath} % disable protrusion for tt fonts
}{}
\usepackage[margin=1in]{geometry}
\usepackage{hyperref}
\hypersetup{unicode=true,
            pdftitle={Introduction},
            pdfauthor={David Bowden},
            pdfborder={0 0 0},
            breaklinks=true}
\urlstyle{same}  % don't use monospace font for urls
\usepackage{longtable,booktabs}
\usepackage{graphicx,grffile}
\makeatletter
\def\maxwidth{\ifdim\Gin@nat@width>\linewidth\linewidth\else\Gin@nat@width\fi}
\def\maxheight{\ifdim\Gin@nat@height>\textheight\textheight\else\Gin@nat@height\fi}
\makeatother
% Scale images if necessary, so that they will not overflow the page
% margins by default, and it is still possible to overwrite the defaults
% using explicit options in \includegraphics[width, height, ...]{}
\setkeys{Gin}{width=\maxwidth,height=\maxheight,keepaspectratio}
\IfFileExists{parskip.sty}{%
\usepackage{parskip}
}{% else
\setlength{\parindent}{0pt}
\setlength{\parskip}{6pt plus 2pt minus 1pt}
}
\setlength{\emergencystretch}{3em}  % prevent overfull lines
\providecommand{\tightlist}{%
  \setlength{\itemsep}{0pt}\setlength{\parskip}{0pt}}
\setcounter{secnumdepth}{5}
% Redefines (sub)paragraphs to behave more like sections
\ifx\paragraph\undefined\else
\let\oldparagraph\paragraph
\renewcommand{\paragraph}[1]{\oldparagraph{#1}\mbox{}}
\fi
\ifx\subparagraph\undefined\else
\let\oldsubparagraph\subparagraph
\renewcommand{\subparagraph}[1]{\oldsubparagraph{#1}\mbox{}}
\fi

%%% Use protect on footnotes to avoid problems with footnotes in titles
\let\rmarkdownfootnote\footnote%
\def\footnote{\protect\rmarkdownfootnote}

%%% Change title format to be more compact
\usepackage{titling}

% Create subtitle command for use in maketitle
\newcommand{\subtitle}[1]{
  \posttitle{
    \begin{center}\large#1\end{center}
    }
}

\setlength{\droptitle}{-2em}
  \title{Introduction}
  \pretitle{\vspace{\droptitle}\centering\huge}
  \posttitle{\par}
  \author{David Bowden}
  \preauthor{\centering\large\emph}
  \postauthor{\par}
  \predate{\centering\large\emph}
  \postdate{\par}
  \date{June 24, 2017}

\usepackage{setspace}

\usepackage{float}
\let\origtable\table
\let\endorigtable\endtable
\renewenvironment{table}[1][2] {
    \singlespacing
    \expandafter\origtable\expandafter[H]
} {
    \endorigtable
}

\frontmatter

\usepackage{amsthm}
\newtheorem{theorem}{Theorem}[chapter]
\newtheorem{lemma}{Lemma}[chapter]
\theoremstyle{definition}
\newtheorem{definition}{Definition}[chapter]
\newtheorem{corollary}{Corollary}[chapter]
\newtheorem{proposition}{Proposition}[chapter]
\theoremstyle{definition}
\newtheorem{example}{Example}[chapter]
\theoremstyle{remark}
\newtheorem*{remark}{Remark}
\begin{document}
\maketitle

\doublespacing

\mainmatter

Why do some civil wars have multiple rebel groups, while others have
only one? Theories of civil war tend to focus on individual- or
group-level motives (e.g. Gurr \protect\hyperlink{ref-gurr70}{1970};
Collier and Hoeffler \protect\hyperlink{ref-Collier2004}{2004}) or
opportunities (e.g. Fearon and Laitin
\protect\hyperlink{ref-fearonlaitin03}{2003}) for rebellion, while
giving little attention to the organization of dissent into rebel groups
and coalitions. Even those studies which do explicitly consider rebel
group formation tend to focus on group attributes such as treatment of
civilians (e.g. Weinstein \protect\hyperlink{ref-Weinstein2007}{2007}),
and do not consider the structure of the rebel movement that emerges.
Yet, at least two rebel groups are simultaneously active at some point
in 44\% of civil conflicts.\footnote{Source: Pettersson and Wallensteen
  (\protect\hyperlink{ref-Pettersson2015a}{2015}).} Over the course of
the Chadian Civil War, for instance, 25 distinct rebel groups appeared
over the course of the conflict. Conflicts in Afghanistan in the 1980's,
Somalia in the 1990's, Sudan in the 2000's have been similarly complex.
The ongoing civil war in Syria is contested by at least two dozen armed
groups. Even ethnically-homogeneous, geographically-concentrated
populations with common goals, such as the Karen secessionist movement
in Myanmar, often fragment into multiple rebel groups. Furthermore, the
number of groups operating in these conflicts often varies greatly over
time. Returning to the Syrian example, the opposition was largely
consolidated under the banner of the Free Syrian Army early in the
conflict, later splintered into dozens of factions largely on the basis
of religion, and now is again reducing in complexity as groups merge or
are defeated.

Several studies examine the implications of rebel movement structure.
Generally, these works find that fragmented rebel movements are
associated with particularly concerning conflict attributes. Conflicts
with multiple rebel groups last longer than dyadic competitions, as the
increased number of veto players complicate the negotiation of peaceful
settlements (Cunningham \protect\hyperlink{ref-Cunningham2006}{2006};
Cunningham, Gleditsch, and Salehyan
\protect\hyperlink{ref-Cunningham2009}{2009}; Akcinaroglu
\protect\hyperlink{ref-Akcinaroglu2012}{2012}), and create the
possibility of peace being spoiled by extreme factions (Stedman
\protect\hyperlink{ref-Stedman1997}{1997}). Relatedly, Cunningham,
Gleditsch, and Salehyan (\protect\hyperlink{ref-Cunningham2009}{2009})
find that the presence of multiple government-rebel dyads decreases the
likelihood that a conflict will end with a peace agreement, while
increasing the likelihood of rebel victory. Findley and Rudloff
(\protect\hyperlink{ref-Findley2012}{2012}) find this effect to be
conditional, however, as the fragmentation of weak rebel movements can
increase the probability of peaceful settlement. Perhaps related to the
paucity of peaceful settlements, both Atlas and Licklider
(\protect\hyperlink{ref-Atlas1999}{1999}) and Zeigler
(\protect\hyperlink{ref-Zeigler2016}{2016}) find that civil wars with
multiple rebel groups are prone to recurrence, as new episodes of
conflict frequently occur between rebel factions from the previous
conflict. Finally, conflicts with multiple dyads feature over 20\% more
fatalities than dyadic ones.\footnote{Source: my own analysis using data
  from Sundberg (\protect\hyperlink{ref-Sundberg2008a}{2008}).} In
short, conflicts with multiple rebel groups are an unusually severe
subset of civil wars.

While prior has firmly established the importance of understanding why
some conflicts have multiple rebel groups while others do not, to date
very few works have attempted to explain this phenomenon. The studies
that do exist in this area tend to focus on a narrow subset of the
processes affecting conflict complexity. For example, several recent
works explore the splintering of existing rebel groups (e.g. McLauchlin
and Pearlman \protect\hyperlink{ref-McLauchlin2012}{2012}; Staniland
\protect\hyperlink{ref-Staniland2014}{2014}). These works tend to focus
on the organizational characteristics of rebel groups however, and thus
have little to say about why rebel groups might form alliances, nor why
entirely new groups might enter a conflict. Christia
(\protect\hyperlink{ref-Christia2012}{2012}) adapts realism from
international relations theory into a unified explanation of splintering
and alliance formation, but she too ignores the mobilization of new
groups. I connect all three phenomena in a single theoretical framework,
providing a unified explanation for conflict complexity. Additionally,
the existing research is largely based on a small number of case studies
disproportionately drawn from the Middle East and South Asia. While
these conflicts are undeniably complex, they are outliers in terms of
both their long duration and high degree of international intervention.
I test my theory on a sample of all civil wars since 1946, demonstrating
that it is widely applicable.

The goal of this project is to address a single broad research question:
what explains the variation in the number of rebel groups in a civil
war? I address several more specific questions in pursuit of this
broader goal. Under what conditions do new rebel groups join an ongoing
civil war? Why do existing rebel groups splinter into multiple factions?
When and why do previously independent rebel groups form alliances?

In brief, I argue that the treatment of civilians during wartime is a
crucial determinant of rebel movement cohesion. Violent repression of
civilians should lead many of them to calculate that joining a rebellion
is not dramatically riskier than remaining non-violent, increasing the
pool of individuals willing to fight. But as repression is often applied
on the basis of ethnicity, and ethnic groups often offer a useful basis
for organizing defensive measures, repression should also tend to induce
greater levels of ethnic identification. Thus, repression should both
create a pool of individuals willing to join the conflict, and lead to
increased demand for rebel groups that emphasize ethnic identity. I
expect that this dynamic will influence all three processes identified
in the existing literature as determinants of conflict complexity ---
the formation of new rebel groups, the splintering of existing rebel
groups, and the merger of previously independent groups into alliances.
The results of my empirical chapters suggest that complex civil wars are
often the result of a sectarian spiral --- an initial wave of repression
mobilizes violent dissent and induces greater levels of ethnic
identification, and the rebel movement fragments along ethnic lines to
reflect these individual-level preferences. Prior work suggests that a
more fragmented movement might lead to greater levels of conflict
severity, closing the vicious circle.

In the remainder of this chapter I review the existing literature on
rebel movement structure, as well as prior work on repression and ethnic
identification. Next, I summarize the broader theoretical and policy
implications of the research. Finally, I provide a summary of the
subsequent chapters.

\section{The Contribution of this
Project}\label{the-contribution-of-this-project}

First and foremost, this project advances our understanding of a subset
of civil wars that is crucially important for the reasons outlined
above. This research explains three processes that account for most of
the variation in the number of rebel groups in a conflict --- the entry
of new groups, and the splintering and mergers of existing ones. This
dissertation is among the first projects to directly address the first
phenomenon of new rebel groups joining ongoing conflicts. Existing work
either considers the formation of new rebel groups (i.e.~groups that
were not previously contained within another violent organization) only
in cases where it is coterminous with conflict initiation (e.g. Lewis
\protect\hyperlink{ref-Lewis2016}{2016}), or in the context of contagion
into previously peaceful areas (e.g. Lane
\protect\hyperlink{ref-Lane2016}{2016}). Yet, I find that 27.5\% of
rebel groups active since World War II were neither present from the
beginning of the conflict, nor is there any evidence that they descended
from existing rebel groups. An important contribution of this
dissertation, then, is explaining this common but mostly ignored
phenomenon.

While splintering and alliance formation have been the subject of
several prior studies, my findings largely contrast with existing work.
Christia (\protect\hyperlink{ref-Christia2012}{2012}) argues that
realist power politics calculations drive both alliance formation and
splintering. By contrast, I find that a rebel group's strength is
predictive of neither its susceptibility to splintering, nor its
propensity to form alliances. Asal, Brown, and Dalton
(\protect\hyperlink{ref-Asal2012}{2012}) and Staniland
(\protect\hyperlink{ref-Staniland2014}{2014}) suggest that
organizational structure is the key determinant of rebel cohesiveness. I
find no evidence for this, however, as I find no evidence that rebel
group centralization is related to splintering. Instead, I find that
both phenomena are strongly related to repression. I also build upon the
existing literature by unifying explanations of splintering and
alliances into a more comprehensive theory of rebel movement structure,
and test my theory in a much wider empirical domain than prior work.

My findings also suggest several important second-order implications.
One is that rebellion seems to be more political and more responsive to
the preferences of rank-and-file dissidents than much of the existing
literature would suggest. A substantial number of scholars view civil
war as largely apolitical, instead being driven by material greed
(Mueller \protect\hyperlink{ref-mueller00}{2000}; Collier and Hoeffler
\protect\hyperlink{ref-Collier2004}{2004}) or personal animosities
(Kalyvas \protect\hyperlink{ref-Kalyvas2006}{2006}). I argue that
dissidents have strong preferences over the content of rebellion, and
find that repression tends to induce stronger preferences for rebel
groups that represent the interests of a particular ethnic group. While
I cannot rule out the possibility that rebellions are initially driven
by material considerations, as war initiation is beyond the scope of
this study, my findings do suggest that civil war violence tends to have
a politicizing effect over time. I discuss the literature on rebel
motives in greater detail later in this chapter.

This work also suggests that governments can exert a powerful influence
on the structure of dissident organizations, as repression can heighten
the salience of identities that divide dissidents. This contrasts with
the most existing accounts of rebel movement structure, which tend to
focus on factors mostly internal to the rebel movement such as relative
power among rebel factions (Christia
\protect\hyperlink{ref-Christia2012}{2012}), or the strength of pre-war
social ties among dissidents (Staniland
\protect\hyperlink{ref-Staniland2014}{2014}). It also raises several
interesting questions about government strategy in the face of dissent.
My findings suggest that repression expands the pool of individuals
willing to fight, which in a vacuum makes government repression
puzzling. It is also unclear whether the other key consequence of
repression I identify --- the increased salience of ethnic identity ---
is a desirable outcome for the government. On one hand it could form the
basis of an effective divide-and-conquer strategy. On the other hand,
fighting multiple opponents could complicate the logistics of
counterinsurgency, threaten the credibility of negotiated settlements,
and undermine the prospects of a stable resolution to the conflict.
While this calculation merits greater consideration that I am able to
give it in this dissertation, my findings in Chapter
\ref{survey-chapter} suggest that repression may be aimed at deterring
dissidents from political activities other than rebellion, such as
voting.

To test my hypotheses, I collected data on the origins of rebel groups.
This allows me to distinguish between groups that splintered from
existing rebel groups, groups engaging in violence for the first time,
and coalitions of previously active groups. I suspect that the causal
factors behind the emergence of each of these types are related, but as
the processes are quite distinct they should be studied separately.
Though I do not make much of the distinctions in this project, the data
also distinguish between several categories of groups that were not
previously engaged in rebellion, including political organizations,
religious organizations, apolitical militias, and factions of the regime
military. These categories should be useful for a variety of future
studies on topics such as the durability of rebel groups, their
probability of victory, and their treatment of civilians.

In addition to resolving a gap in the scholarly literature, a better
understanding of rebel movement structure is of value to policymakers.
As noted above, conflicts with multiple rebel groups are among the most
severe. Simply being able to predict which conflicts are likely to
become severe through this mechanism has several useful applications.
Policymakers might be able to identify early on the conflicts that are
most likely to benefit from peace operations. Humanitarian organizations
could predict which conflicts are likely to produce large numbers of
refugees, and distribute resources accordingly. This work also be the
possibility of moving beyond prediction and solving the underlying
problem. As the empirical analyses identify the repression of civilians
as a key mechanism driving conflict complexity, it stands to reason that
protecting civilians might be an especially valuable undertaking for
non-governmental organizations or outside states.

In the next section I situate my dissertation in the literature to which
it is most closely related. I explain in greater detail my contributions
over the existing work on rebel structure, and also discuss the
implications of this work for the literatures on repression and ethnic
identification.

\section{Previous Work on the Organization of
Rebellion}\label{previous-work-on-the-organization-of-rebellion}

The existing literature and empirical record suggest that the number of
rebel groups active in a conflict is shaped by three broad processes.
New groups can emerge when previously non-violent individuals mobilize
and join the conflict. Alternatively, previously cohesive rebel groups
can splinter into multiple successor organizations. Finally, the number
of rebel groups can decrease when previously independent factions form
alliances. I summarize the literature on each process in turn, and
relate my contributions to the existing work.

\subsection{Group Formation}\label{group-formation}

Around 30\% of conflicts have at least one rebel group that was neither
active from its beginning, nor did it split from an existing rebel
group. Yet few studies directly consider the phenomenon of new rebel
groups joining ongoing conflicts. Even studies of civil war onset often
leave the formation of rebel groups in a black box, instead making a
leap from individual motives to war initiation. For instance, a large
literature views rebellion as an essentially criminal activity, driven
by greed (Mueller \protect\hyperlink{ref-mueller00}{2000}; Collier and
Hoeffler \protect\hyperlink{ref-Collier2004}{2004}; Lujala, Gleditsch,
and Gilmore \protect\hyperlink{ref-Lujala2005}{2005}). Yet these works
generally have very little to say about the origins of rebel
organizations. These groups could be pre-existing criminal organizations
that initiate more violent activity in hopes of securing greater profit,
they could form for the purpose of a greed-driven rebellion after a sign
of weakness from the government, or they could begin as rebel groups
with sincere political goals, which are later seduced into less noble
pursuits. The grievance school similarly tends to neglect group
formation. For example, Cederman, Wimmer, and Min
(\protect\hyperlink{ref-Cederman2010}{2010}) offer a nuanced explanation
of the conditions under which ethnic minorities are likely to rebel.
Yet, they say little about the logistics of organizing a rebellion, and
seemingly assume that ethnic groups have an inherent ability to spawn
rebel organizations.

Scholars working at lower levels of analysis have come closer to
explaining group formation. Kalyvas
(\protect\hyperlink{ref-Kalyvas2006}{2006}) suggests that individuals
are often already mobilized for small-scale violence such as personal
rivalry, criminal activity, or ethnic conflict. Building a rebel group
is thus an exercise in building coalitions from small, pre-existing
organizations, and re-orienting individuals from localized issues to
national-level political cleavages. Kalyvas gives little attention to
this process, however, instead recommending it as an area for future
research. Staniland (\protect\hyperlink{ref-Staniland2014}{2014}) also
argues that rebel groups can trace their origins to pre-existing social
organizations, though he sees larger, and often more political entities
such as political parties or military units as the primary source of
rebellion, rather than the localized and less formal groups emphasized
by Kalyvas (\protect\hyperlink{ref-Kalyvas2006}{2006}). Staniland
(\protect\hyperlink{ref-Staniland2014}{2014}) too devotes relatively
little space to rebel group formation, instead focusing on linking the
attributes of the originating organizations to rebel group outcomes such
as durability. Lewis (\protect\hyperlink{ref-Lewis2016}{2016}) offers a
somewhat contrasting view as she carefully documents the earliest
activities of rebel groups in Uganda. She finds that rebel groups,
including the Lord's Resistance Army, were typically founded by small
number of entrepreneurial individuals, and initially tended to value
stealth over broad mobilization. Only after the conflict began to
escalate did groups seek to broaden their membership, in many cases by
appealing to particular ethnic groups. Thus she sees scholars such as
Cederman, Wimmer, and Min (\protect\hyperlink{ref-Cederman2010}{2010})
and Staniland (\protect\hyperlink{ref-Staniland2014}{2014}) as beginning
their analyses after rebellion had existed for some time.

There is also a substantial literature on the contagion of civil war.
For example, Gleditsch (\protect\hyperlink{ref-Gleditsch2007}{2007})
finds that transnational ethnic groups and political and economic
linkages between states can provide channels for civil war to spread
across international boundaries. Most of his cases, however, are
pre-existing rebel groups moving into new geographic areas, rather than
\emph{sui generis} group formation. Other scholars find that entirely
new rebel organizations can emerge through the contagion of secessionist
(Ayres and Saideman \protect\hyperlink{ref-Ayres2000}{2000}) and ethnic
(Lane \protect\hyperlink{ref-Lane2016}{2016}) conflict. Such
transnational processes might shape opportunities for multiple
rebellions to emerge by increasing the availability of weapons,
spreading tactical knowledge, or diverting government attention to
foreign conflicts. While contagion explains an important category of
phenomena, these studies are primarily concerned with the spread of
conflict to previously peaceful areas. This overlaps only partly with
the scope of this project; I am also concerned with the emergence of new
rebel groups in areas already experiencing conflict.

In short, surprisingly few studies have given much consideration to the
formation of rebel groups. The few that do (e.g. Lewis
\protect\hyperlink{ref-Lewis2016}{2016}) focus entirely on groups whose
origins coincide with war onset. While research on contagion effects
sheds light on the expansion of conflict, it does not address the entry
of new groups to existing conflict zones. I seek to resolve this gap.

\subsection{Splintering}\label{splintering}

Existing rebel groups frequently splinter into multiple successor
organizations. In 1968, for example, a faction led by Ahmed Jibril broke
away from the Popular Front for the Liberation of Palestine (PFLP) to
form a new group, the Popular Front for the Liberation of
Palestine-General Command (PFLP-GC). While the two groups often
collaborated against Israel, they maintain distinct organizational
structures and membership bases, and operate in different areas. The
split was allegedly motivated by differing views of Marxist ideology and
military doctrine, with the PFLP pursuing a more extreme strategy of
attrition. Similar splits have occurred within dozens of rebel groups,
including the Communist Party of Burma, the Free Syrian Army and the
Sudan Liberation Army. In many cases the result is more than a nominal
separation. In Sri Lanka, for example, the Tamil Peoples Liberation
Tigers not only split from the Liberation Tigers of Tamil Eelam, but
also defected to the government side in the conflict (Staniland
\protect\hyperlink{ref-Staniland2012d}{2012}).

Compared to group formation, there is a relatively large literature on
rebel group splintering. One subset of this research focuses on the role
of external actors, and particularly the government. For instance,
McLauchlin and Pearlman (\protect\hyperlink{ref-McLauchlin2012}{2012})
find that government repression provides occasion for groups to evaluate
their current leadership structure. Pre-existing divisions within groups
are likely to be exacerbated, leading the group to move toward more
factionalized leadership structures. When group members are satisfied,
however, conflict tends to lead to even greater unity and centralization
of authority. Whereas the preceding studies essentially treat government
repression as exogenous to the internal politics of dissident groups,
Bhavnani, Miodownik, and Choi
(\protect\hyperlink{ref-Bhavnani2011}{2011}) present evidence that
governments deliberately stoke tensions among their opponents, as they
find that the Israeli government increased conflict between Fatah and
Hamas by undermining Hamas' control of the Gaza and by tolerating
Fatah's relationship with the Jordanian military. Tamm
(\protect\hyperlink{ref-Tamm2016}{2016}) finds that support from outside
states can alter the balance of power within rebel groups, in some cases
entrenching existing hierarchies, while in others creating possibilities
for fragmentation or coups. Finally, Staniland
(\protect\hyperlink{ref-Staniland2012d}{2012}) finds that the government
can sometimes attract rebel groups to their side by offering greater
resources during periods of infighting among rebel groups.

Another group of scholars emphasizes concerns about post-conflict
bargaining as the key determinant of dissident group cohesion. Christia
(\protect\hyperlink{ref-Christia2012}{2012}) assumes that the winning
coalition in a civil war receives private benefits, which might include
any rents available to the state, and having some portion of its
interests represented in the new government. Thus, rebels have an
incentive to form minimum winning coalitions, so as to limit the number
of coalition partners with whom they must share benefits. Wolford,
Cunningham, and Reed (\protect\hyperlink{ref-Wolford}{2015}) develop a
similar logic, theorizing that political factions have an interest in
joining conflicts so as to maximize the likelihood of their preferences
being represented in the post-war government, but the value of fighting
decreases as the number of parties with whom they expect to share power
increases. Yet, Christia (\protect\hyperlink{ref-Christia2012}{2012})
suggests that this incentive to minimize coalition size is moderated by
the risk of being outside the winning coalition, as there is a strong
possibility of new waves of violence between victorious rebels and rival
rebel factions. She thus expects coalitions to change frequently in
response to battlefield events, with factions bandwagoning with battle
winners and shifting away from losing coalitions. Findley and Rudloff
(\protect\hyperlink{ref-Findley2012}{2012}) similarly find fragmentation
to be most common among groups that have recently lost battles. This
implies that fragmentation is essentially a process of weak actors
becoming weaker.

A final category of explanations places the source of rebel group
cohesion in underlying social structure. Staniland
(\protect\hyperlink{ref-Staniland2014}{2014}) argues that insurgent
organizations will be most stable when their central leadership is able
to exercise both vertical control over its rank-and-file members, and
horizontal control over its constituent groups. This is most likely to
occur when insurgencies draw from existing organizations with extant
social ties of this sort, which might include former anti-colonial
movements or ethnic political parties. Organizations are likely to
fragment when constituent groups have a high degree of autonomy or
control over individual members is limited (Staniland
\protect\hyperlink{ref-Staniland2014}{2014}, Ch. 2-3). Asal, Brown, and
Dalton (\protect\hyperlink{ref-Asal2012}{2012}) emphasize similar
factors, arguing that organizations with factionalized leadership
structures are at risk of fragmentation, while groups with more
consolidated power structures will tend to remain cohesive. Finally,
Warren and Troy (\protect\hyperlink{ref-Warren2015}{2015}) suggest that
group size plays an important role, as small groups are able to police
themselves and resolve conflicts, whereas larger groups are more likely
to experience infighting.

The existing work in this field tends to feature impressive data
collection or fieldwork, and makes important contributions to our
understanding of rebel group cohesion. Yet, making predictions from
existing approaches tends to require detailed information about a rebel
group and its internal workings, which is difficult to acquire,
particularly while a conflict is still active. Furthermore, this
literature tends to be somewhat disconnected from work on other rebel
attributes and behaviors, including alliance formation. This
dissertation addresses these limitations by unifying the formation of
new groups, splintering, and alliance formation under a single
theoretical framework which relies on explanatory factors that are
relatively easy to observe, allowing for predictions about rebel group
structure even during conflicts.

\subsection{Alliance Formation}\label{alliance-formation}

Empirically, alliances among rebel groups are both common and
noteworthy. Many of the most successful rebel movements in history were
coalitions of formerly independent organizations. For example, the
Frente Farabundo Marti para la Liberacion Nacional (FMLN) was an
umbrella organization uniting several left wing rebel groups in El
Salvador, which eventually secured many concessions in the post-war
peace process including a place as a major political party. Surpsingly,
however, alliances among non-state actors have only recently begun to
receive much scholarly attention.

Asal and Rethemeyer (\protect\hyperlink{ref-Asal2008}{2008}) and
Horowitz and Potter (\protect\hyperlink{ref-Horowitz2013}{2013}) conduct
network analyses of alliance formation among terrorist groups, arguing
that such arrangement are used to aggregate capabilities and share
tactics. Much of the other work in the field focuses on the downsides of
alliance. Bapat and Bond (\protect\hyperlink{ref-Bapat2012}{2012})
assume that alliances carry two significant costs: the dilution of each
constituent group's agenda, and the risk of having one's private
information sold to the government by an ally. Consistent with this
theory, they find alliances to be most common when an outside state can
enforce agreements, and when all rebel groups involved are strong enough
to avoid the temptation of defecting to the government side. Christia
(\protect\hyperlink{ref-Christia2012}{2012}) similarly emphasizes
capability, arguing that neorealist balancing theory from international
relations explains alignments in civil wars. When one coalition - a
group of rebels or government-aligned forces - becomes too powerful,
other groups will band together to prevent their own destruction. But
similar to Bapat and Bond (\protect\hyperlink{ref-Bapat2012}{2012}),
Christia (\protect\hyperlink{ref-Christia2012}{2012}) argues that this
mechanism is constrained by a desire to maximize one's share of the
post-war spoils. Thus, rebels realign frequently, seeking to form
minimum winning coalitions. While shared identity appears on the surface
to be an important determinant of rebel alignments, Christia views these
narratives as post-hoc justifications aimed at legitimizing decisions
that are really driven mostly by power.

The existing studies do much to advance our knowledge of relationships
between non-state actors. Yet, some important aspects of alliance
formation are beyond the scope of the existing studies, however. Namely,
while relative power considerations can potentially account for why
rebel groups form alliances, and when they will alter their ties, it
does not explain why groups choose a particular partner. Horowitz and
Potter (\protect\hyperlink{ref-Horowitz2013}{2013}) do shed light on
this question, finding that militants prefer to ally with powerful
groups, but their focus is largely on transnational networks of
terrorists and insurgents, rather than alliance formation within a
particular conflict. I offer an explanation of alliance formation that
can predict not only when rebel groups will seek alliances, but also
with whom. Furthermore, I connect this process to the broader dynamics
of rebel movement structure, particularly splintering.

\subsection{Repression}\label{repression}

As repression is central to the theoretical argument presented here,
this dissertation is shaped by and contributes to the literature on the
topic. The focus in the existing literature has been on explaining why
repression occurs, and identifying factors that might prevent it.
Davenport (\protect\hyperlink{ref-Davenport2007a}{2007}) finds that
there is a ``domestic democratic peace,'' meaning that democratic
regimes tend to refrain from using the most violent forms of repression.
However, he finds that even democracies often engage in repression
during civil and international conflicts. Others find that international
human rights treaties often have a meaningful restraining effect on
governments, reducing their use of repression (Hathaway
\protect\hyperlink{ref-Hathaway2002}{2002}; Simmons
\protect\hyperlink{ref-simmons09}{2009}). Not all international
influences are positive, however, as economic sanctions (R. M. Wood
\protect\hyperlink{ref-Wood2008a}{2008}) are associated with increased
repression. An important generalization in the context of this study is
that human rights practices tend to be shaped by domestic and
international political institutions that are likely to be largely
exogenous to civil war dynamics.\footnote{Long-running civil wars,
  however, might deter democratization and participation in human rights
  treaties.}

Another strand of the repression literature focuses on the consequences
of repression, and especially the potential of repression to provoke
escalation. In this vein Lichbach
(\protect\hyperlink{ref-Lichbach1987}{1987}) argues that repression
should lead dissidents to substitute increasingly violent tactics for
more peaceful ones, as they will calculate that violence is more likely
to achieve their goals. Moore (\protect\hyperlink{ref-Moore1998}{1998})
finds empirical support for this model, suggesting that repression has
significant potential to escalate political confrontations. My findings
in Chapter \ref{survey-chapter} add further support for this model, as I
show that repression increases willingness to engage in violence, while
decreasing the likelihood that individuals will engage in peaceful
political activities such as voting. I also move beyond this debate to
show that repression can shape not only the likelihood, but also the
structure of civil war violence.

\subsection{Ethnic Identification}\label{ethnic-identification}

Ethnic identity is central to the theoretical mechanism in this
dissertation, and has long been an area of deep interest to scholars of
comparative politics. This work is often predicated on the assumption
that identity is dynamic. At a minimum, individuals can choose which of
their several social roles to emphasize. For instance, individuals might
orient primarily toward an ethnicity, a religion, an occupation, a
region, or an ideology, and could potentially alter these choices over
time. The majority of the work in this vein has focused on oscillations
between ethnic and national identities. Early on this question was
explored in discussions of statebuilding. Scholars in this area suggest
that external threats such as interstate wars (Herbst
\protect\hyperlink{ref-Herbst1990}{1990}; Tilly
\protect\hyperlink{ref-Tilly1992}{1992}) or territorial disputes
(Gibler, Hutchison, and Miller \protect\hyperlink{ref-Gibler2012}{2012})
can provide a unifying influence, leading individuals to orient toward
national identities and away from subnational ones such as ethnicity.
Most other work in the area examines the role of political institutions
in incentivizing the use of particular identities (Posner
\protect\hyperlink{ref-Posner2005}{2005}; Penn
\protect\hyperlink{ref-Penn2008}{2008}). For example, Eifert, Miguel,
and Posner (\protect\hyperlink{ref-Eifert2010}{2010}) find that ethnic
identification tends to be strongest just prior to or just after
competitive elections, suggesting that individuals interpret politics
through an ethnic lens.

A striking feature of this literature is the degree of consensus that
ethnic identity is malleable. This perspective is shared by a diverse
range of scholars ranging from constructivists (e.g. Barnett
\protect\hyperlink{ref-Barnett1995}{1995}) to formal theorists (e.g.
Penn \protect\hyperlink{ref-Penn2008}{2008}), and enjoys strong
empirical support (e.g. Eifert, Miguel, and Posner
\protect\hyperlink{ref-Eifert2010}{2010}; Gibler, Hutchison, and Miller
\protect\hyperlink{ref-Gibler2012}{2012}). Thus, it provides a strong
foundation on which to build my theory. This dissertation also
contributes to the ethnic identification in two ways. First, while
previous work has suggested that internal conflict might have an
opposite effect to external wars, leading to increased ethnic
identification (Kaufmann \protect\hyperlink{ref-Kaufmann1996b}{1996}),
no existing work has demonstrated this systematically. My dissertation
begins to resolve this gap, as Chapter \ref{survey-chapter} shows that
both repression and the presence of civil wars are associated with
heightened levels of ethnic identification. Second, whereas most prior
studies examine individual-level ethnic identification in isolation
(e.g. Eifert, Miguel, and Posner
\protect\hyperlink{ref-Eifert2010}{2010}), I connect the phenomenon to
aggregate outcomes, namely the formation and restructuring of rebel
groups.

\subsection{Rebel Motives}\label{rebel-motives}

An examination of the relationships between dissident groups is also
likely to offer a new perspective on rebel motives. The literature on
civil war has largely been dominated by debates over whether rebellion
is fundamentally political, or done in pursuit of private benefits. The
former views civil war as an effort to resolve economic or political
inequality (Gurr \protect\hyperlink{ref-gurr70}{1970}; E. J. Wood
\protect\hyperlink{ref-Wood2003}{2003}; Cederman, Wimmer, and Min
\protect\hyperlink{ref-Cederman2010}{2010}), and has been labeled as the
`grievance' hypothesis (Collier and Hoeffler
\protect\hyperlink{ref-Collier2004}{2004}). The latter is composed
primarily of studies emphasizing the `greed' hypothesis (Collier and
Hoeffler \protect\hyperlink{ref-Collier2004}{2004}), which views
rebellion as little more than large-scale criminal activity aimed at
bringing profits to its members (Mueller
\protect\hyperlink{ref-mueller00}{2000}; Lujala, Gleditsch, and Gilmore
\protect\hyperlink{ref-Lujala2005}{2005}; Ross
\protect\hyperlink{ref-Ross2004e}{2004}). Others have emphasized
non-material private benefits as motive for individual participation in
rebellion, such as the ability to act on family disputes or romantic
rivalries (Kalyvas \protect\hyperlink{ref-Kalyvas2006}{2006}).

This political-private motive debate has yet to be definitively
resolved. A number of scholars have found greater support for the greed
hypothesis than for grievance, with the presence of natural resources
being a stronger predictor of civil war than economic or political
grievances (Collier and Hoeffler
\protect\hyperlink{ref-Collier2004}{2004}). Yet, these findings are not
robust across different types of resources or even different measures of
the same resource (Dixon \protect\hyperlink{ref-Dixon2009a}{2009}).
Furthermore, several scholars have found that political factors such as
hierarchical relationships between ethnic groups (Cederman, Wimmer, and
Min \protect\hyperlink{ref-Cederman2010}{2010}) and poor economic
performance (Miguel, Satyanath, and Sergenti
\protect\hyperlink{ref-Miguel2004a}{2004}) exert a strong influence on
civil war onset. Other scholars eschew the dichotomy altogether,
suggesting that while private benefits are useful to rebel recruiting
efforts, this does not preclude the possibility that rebel elites
ultimately have political motives (Lichbach
\protect\hyperlink{ref-Lichbach1995}{1995}; Weinstein
\protect\hyperlink{ref-Weinstein2007}{2007}). Similarly, Lujala
(\protect\hyperlink{ref-Lujala2010}{2010}) finds that natural resources
are associated with longer conflicts, implying that at least a portion
of resource revenues are devoted to fighting rather than private
benefits.

One factor that has limited progress on these questions of motive is the
fact that the competing theories have been tested almost exclusively on
a single outcome --- a binary measure of the occurrence of civil war at
the national level. Studying the relationships between dissident groups
and how they vary is likely to provide insight to underlying rebel
motives. For instance, if rebellion is fundamentally about maximizing
the profits of its members, we might expect to see rebels form the
smallest coalitions possible that still allow them to control resource
flows. If rebellion is fundamentally political, however, we might expect
rebels to pursue coalitions large enough to pursue victory.
Additionally, if ideology and identity are not truly important to
rebels, splintering and alliance formation should be driven primarily by
power calculations (see Christia
\protect\hyperlink{ref-Christia2012}{2012}). If these factors do matter,
however, they should shape the choice of alliance partners and the
cohesiveness of individual groups. Ethnically homogenous rebel groups
should be less prone to splintering in this case, and alliance should be
more likely among groups with similar identities.

\section{Project Summary}\label{project-summary}

In Chapter \ref{theory} I articulate a theory of rebel movement
structure. I begin with the assumption that rebel groups emerge from a
broader pool of dissidents. While not all dissidents will be eager to
participate in violence, each will prefer to be represented by a rebel
group which advances their political interests and provides them with
security. Thus, dissidents form a constituency that constantly evaluates
the performance of rebel groups, and will consider switching their
allegiance to new groups if the existing ones are lacking. In hopes of
seizing on this dynamic, rebel entrepreneurs will look for opportunities
to mobilize new groups by appealing to underrepresented identities and
ideologies. These appeals should be especially effective in the wake of
repression for two reasons. First, repression lowers the risk of
fighting relative to remaining peaceful, leading new individuals to join
the fighting. Second, as repression should induce greater levels of
ethnic identification. Repression is often targeted on the basis of
ethnicity, increasing its salience, and appeals for support from
outside, co-ethnic states might be especially effective in the presence
of human rights concerns. Thus repression not only creates a new pool of
individuals willing to fight, it also stokes division among dissidents
along ethnic lines. This should often lead individuals joining the
fighting to form new groups rather than join existing ones, and
individuals already in rebel groups to realign into more ethnically
homogeneous configurations. This should manifest in the form of both the
splintering of existing groups, particularly when existing groups are
multi-ethnic, and the formation of ethnically-homogeneous alliances so
as to replace the loss in capabilities due to fragmentation and
streamline access to support from co-ethnic outside states.

Chapter \ref{survey-chapter} tests the individual-level assumptions of
the theory. Using a sample of over 150,000 Afrobarometer Survey
responses, I find support for both of my key predictions regarding the
effects of repression. Individuals who have experienced an attack in the
past year are 30\% more likely than others to express willingness to use
violence themselves. Additionally, I find these individuals are 62\%
more likely to identify with their ethnic group than respondents who
have not experienced an attack. While I am unable to completely rule out
the possibility of reverse causality, the results hold after performing
coarsened exact matching, showing that attacked individuals do not
systematically differ from others on observable traits. These results
suggest that my theory performs as expected at the individual level.

Chapter \ref{entry} contains tests of my predictions regarding the
formation of new rebel groups during ongoing conflicts. I add my measure
of rebel group origin to the Uppsala Armed Conflict Dataset, resulting
in a sample of all civil wars, 1946--2015. I find that the probability
of a new rebel group joining an ongoing conflict during a given year has
a strong, negative relationship with changes in respect for human
rights. The largest observed increases in repression are associated with
more than a 60\% chance of new rebel groups forming, while the
probability is around 3\% in years with no substantial change in human
rights practices, and approaches zero in following improvements to human
rights. Contrary to my expectations, I find no evidence that ethnic
diversity places a scope condition on my theory; new rebel groups form
in a variety of societies. I also do not find evidence that rebels
groups which join ongoing conflicts are more likely than others to draw
their support from a single ethnic group. However, this result seems to
be driven by a large number of rebel groups with no discernible ties to
an ethnic constituency, and I do find that these joining rebel are
significantly less likely than others to be multi-ethnic coalitions. To
supplement these quantitative analyses, I illustrate the causal logic of
my theory in a qualitative case study of the Mon separatists in Burma,
and use the emergence of the All Burma Students' Democratic Front to
explore the limits of my argument.

In Chapter \ref{realignment} I explore two processes through which
rebels reorganize --- splintering from existing groups, and the
formation of alliances. I find that increases in repression are
associated with an increased probability of rebel group splintering,
though the result is not entirely robust. I do not find evidence to
support my hypothesis that rebel groups which draw support from multiple
ethnic groups are more prone to fragmentation. This seems to largely
reflect the fact that ethnically-homogeneous groups are
disproportionately likely to fight long-lasting, low-intensity
separatist conflicts. I illustrate both findings with a study of the
Karen National Union, which originally split from the multi-ethnic
Burmese independence movement to advance the interests of the Karen
people, but later splintered itself along religious lines. Finally,
consistent with my expectations I find evidence that repression
increases the probability that new ethnically-homogenous alliances will
form, while having no effect on the formation multi-ethnic alliances.
Though less robust than the findings in previous chapters, these results
suggest that repression can initiate a process of realignment whereby
rebels tend to leave multi-ethnic coalitions and form new alliances
centered around a particular ethnic identity.

Finally, I summarize the results in Chapter \ref{conclusion}, discuss
their theoretical and policy significance, and propose several avenues
for future research on this topic.

\chapter*{References}\label{references}
\addcontentsline{toc}{chapter}{References}

\markboth{REFERENCES}{}

\indent

\setlength{\parindent}{-0.2in} \setlength{\leftskip}{0.2in}
\setlength{\parskip}{8pt}

\singlespacing

\hypertarget{refs}{}
\hypertarget{ref-Akcinaroglu2012}{}
Akcinaroglu, Seden. 2012. ``Rebel Interdependencies and Civil War
Outcomes.'' \emph{Journal of Conflict Resolution} 56 (5): 879--903.

\hypertarget{ref-Asal2008}{}
Asal, Victor, and R. Karl Rethemeyer. 2008. ``The Nature of the Beast:
Organizational Structures and the Lethality of Terrorist Attacks.''
\emph{The Journal of Politics} 70 (2): 437--49.

\hypertarget{ref-Asal2012}{}
Asal, Victor, Mitchell Brown, and Angela Dalton. 2012. ``Why Split?
Organizational Splits among Ethnopolitical Organizations in the Middle
East.'' \emph{Journal of Conflict Resolution} 56 (1): 94--117.

\hypertarget{ref-Atlas1999}{}
Atlas, Pierre M., and Roy Licklider. 1999. ``Conflict Among Former
Allies After Civil War Settlement : Sudan , Zimbabwe , Chad , and
Lebanon.'' \emph{Journal of Peace Research} 36 (1): 35--54.

\hypertarget{ref-Ayres2000}{}
Ayres, R. William, and Stephen Saideman. 2000. ``Is separatism as
contagious as the common cold or as cancer? Testing international and
domestic explanations.'' \emph{Nationalism and Ethnic Politics} 6 (3):
91--113.

\hypertarget{ref-Bapat2012}{}
Bapat, Navin, and Kanisha Bond. 2012. ``Alliances between Militant
Groups.'' \emph{British Journal of Political Science} 42 (4): 793--824.

\hypertarget{ref-Barnett1995}{}
Barnett, Michael N. 1995. ``Sovereignty, nationalism, and regional order
in the Arab states system.'' \emph{International Organization} 49 (3):
479.

\hypertarget{ref-Bhavnani2011}{}
Bhavnani, Ravi, Dan Miodownik, and Hyun Jin Choi. 2011. ``Three Two
Tango: Territorial Control and Selective Violence in Israel, the West
Bank, and Gaza.'' \emph{Journal of Conflict Resolution} 55 (1): 133--58.

\hypertarget{ref-Cederman2010}{}
Cederman, Lars-Erik, Andreas Wimmer, and Brian Min. 2010. ``Why do
ethnic groups rebel?: New data and analysis.'' \emph{World Politics} 62
(1): 87--98.

\hypertarget{ref-Christia2012}{}
Christia, Fotini. 2012. \emph{Alliance Formation in Civil Wars}.
Cambridge: Cambridge University Press.

\hypertarget{ref-Collier2004}{}
Collier, Paul, and Anke Hoeffler. 2004. ``Greed and grievance in civil
war.'' \emph{Oxford Economic Papers} 56 (4): 563--95.

\hypertarget{ref-Cunningham2006}{}
Cunningham, David E. 2006. ``Veto Players and Civil War Duration.''
\emph{American Journal of Political Science} 50 (4): 875--92.

\hypertarget{ref-Cunningham2009}{}
Cunningham, David E., Kristian Skrede Gleditsch, and Idean Salehyan.
2009. ``It Takes Two: A Dyadic Analysis of Civil War Duration and
Outcome.'' \emph{Journal of Conflict Resolution} 53 (4): 570--97.

\hypertarget{ref-Davenport2007a}{}
Davenport, Christian. 2007. \emph{State Repression and the Domestic
Democratic Peace}. Cambridge: Cambridge University Press.

\hypertarget{ref-Dixon2009a}{}
Dixon, Jeffrey. 2009. ``What Causes Civil Wars? Integrating Quantitative
Research Findings.'' \emph{International Studies Review} 11 (4):
707--35.

\hypertarget{ref-Eifert2010}{}
Eifert, Benn, Edward Miguel, and Daniel N. Posner. 2010. ``Political
competition and ethnic identification in Africa.'' \emph{American
Journal of Political Science} 54 (2): 494--510.

\hypertarget{ref-fearonlaitin03}{}
Fearon, James D., and David D. Laitin. 2003. ``Ethnicity, Insurgency,
and Civil War.'' \emph{American Political Science Review} 97 (1):
75--90.

\hypertarget{ref-Findley2012}{}
Findley, Michael, and Peter Rudloff. 2012. ``Combatant Fragmentation and
the Dynamics of Civil Wars.'' \emph{British Journal of Political
Science} 42 (4): 879--901.

\hypertarget{ref-Gibler2012}{}
Gibler, Douglas M., Marc L. Hutchison, and Steven V. Miller. 2012.
``Individual Identity Attachments and International Conflict: The
Importance of Territorial Threat.'' \emph{Comparative Political Studies}
45 (12): 1655--83.

\hypertarget{ref-Gleditsch2007}{}
Gleditsch, Kristian Skrede. 2007. ``Transnational Dimensions of Civil
War.'' \emph{Journal of Peace Research} 44 (3): 293--309.

\hypertarget{ref-gurr70}{}
Gurr, Ted Robert. 1970. \emph{Why Men Rebel}. Princeton, NJ: Princeton
University Press.

\hypertarget{ref-Hathaway2002}{}
Hathaway, Oona A. 2002. ``Articles Do Human Rights Treaties Make a
Difference ?'' \emph{Yale Law Journal} 111 (8): 1935--2042.

\hypertarget{ref-Herbst1990}{}
Herbst, Jeffrey. 1990. ``War and the State in Africa.''
\emph{International Security} 14 (4): 117--39.

\hypertarget{ref-Horowitz2013}{}
Horowitz, Michael C., and Philip B. K. Potter. 2013. ``Allying to Kill:
Terrorist Intergroup Cooperation and the Consequences for Lethality.''
\emph{Journal of Conflict Resolution} 58 (2): 199--225.

\hypertarget{ref-Kalyvas2006}{}
Kalyvas, Stathis N. 2006. \emph{The Logic of Violence in Civil War}.
Cambridge: Cambridge University Press.

\hypertarget{ref-Kaufmann1996b}{}
Kaufmann, Chaim. 1996. ``Possible and Impossible Solutions to Ethnic
Civil Wars.'' \emph{International Security} 20 (4): 136--75.

\hypertarget{ref-Lane2016}{}
Lane, Matthew. 2016. ``The Intrastate Contagion of Ethnic Civil War.''
\emph{Journal of Politics} 78 (2): 1--15.

\hypertarget{ref-Lewis2016}{}
Lewis, Janet I. 2016. ``How Does Ethnic Rebellion Start ?''
\emph{Comparative Political Studies}, forthcoming.

\hypertarget{ref-Lichbach1987}{}
Lichbach, Mark Irving. 1987. ``Deterrence or Escalation? The Puzzle of
Aggregate Studies of Repression and Dissent.'' \emph{Journal of Conflict
Resolution} 31 (2): 266--97.

\hypertarget{ref-Lichbach1995}{}
---------. 1995. \emph{The Rebel's Dilemma}. Ann Arbor, MI: University
of Michigan Press.

\hypertarget{ref-Lujala2010}{}
Lujala, Päivi. 2010. ``The spoils of nature: Armed civil conflict and
rebel access to natural resources.'' \emph{Journal of Peace Research} 47
(1): 15--28.

\hypertarget{ref-Lujala2005}{}
Lujala, Päivi, Nils Petter Gleditsch, and Elisabeth Gilmore. 2005. ``A
Diamond Curse?: Civil War and a Lootable Resource.'' \emph{Journal of
Conflict Resolution} 49 (4): 538--62.

\hypertarget{ref-McLauchlin2012}{}
McLauchlin, Theodore, and Wendy Pearlman. 2012. ``Out-Group Conflict,
In-Group Unity?: Exploring the Effect of Repression on Intramovement
Cooperation.'' \emph{Journal of Conflict Resolution} 56 (1): 41--66.

\hypertarget{ref-Miguel2004a}{}
Miguel, Edward, Shanker Satyanath, and Ernest Sergenti. 2004. ``Economic
Shocks and Civil Conflict: An Instrumental Variables Approach.''
\emph{Journal of Political Economy} 112 (4): 725--53.

\hypertarget{ref-Moore1998}{}
Moore, Will H. 1998. ``Repression and Dissent: Substitution, Context,
and Timing.'' \emph{American Journal of Political Science} 42 (3):
851--73. doi:\href{https://doi.org/10.2307/2991732}{10.2307/2991732}.

\hypertarget{ref-mueller00}{}
Mueller, John. 2000. ``The Banality of Ethnic War.'' \emph{International
Security} 25 (1): 42--70.

\hypertarget{ref-Penn2008}{}
Penn, Elizabeth Maggie. 2008. ``Citizenship versus Ethnicity: The Role
of Institutions in Shaping Identity Choice.'' \emph{The Journal of
Politics} 70 (4): 956--73.

\hypertarget{ref-Pettersson2015a}{}
Pettersson, Therése, and Peter Wallensteen. 2015. ``Armed conflicts,
1946-2014.'' \emph{Journal of Peace Research} 52 (4): 536--50.

\hypertarget{ref-Posner2005}{}
Posner, Daniel N. 2005. \emph{Institutions and Ethnic Politics in
Africa}. Cambridge: Cambridge University Press.

\hypertarget{ref-Ross2004e}{}
Ross, Michael L. 2004. ``How Do Natural Resources Influence Civil War?
Evidence from Thirteen Cases.'' \emph{International Organization} 58
(01): 35--67.

\hypertarget{ref-simmons09}{}
Simmons, Beth A. 2009. \emph{Mobilizing for Human Rights: International
Law in Domestic Politics}. Cambridge: Cambridge University Press.

\hypertarget{ref-Staniland2012d}{}
Staniland, Paul. 2012. ``Between a Rock and a Hard Place: Insurgent
Fratricide, Ethnic Defection, and the Rise of Pro-State
Paramilitaries.'' \emph{Journal of Conflict Resolution} 56 (1): 16--40.

\hypertarget{ref-Staniland2014}{}
---------. 2014. \emph{Networks of Rebellion: Explaining Insurgent
Cohesion and Collapse}. Ithaca, NY: Cornell University Press.

\hypertarget{ref-Stedman1997}{}
Stedman, Stephen John. 1997. ``Spoiler Problems in Peace Processes.''
\emph{International Security} 22 (2): 5.
doi:\href{https://doi.org/10.2307/2539366}{10.2307/2539366}.

\hypertarget{ref-Sundberg2008a}{}
Sundberg, Ralph. 2008. ``Collective Violence 2002-2007: Global and
Regional Trends.'' In \emph{States in Armed Conflict 2007}, edited by
Lotta Harbom and Ralph Sundberg. Uppsala: Universitetstryckeriet.

\hypertarget{ref-Tamm2016}{}
Tamm, Henning. 2016. ``Rebel Leaders, Internal Rivals, and External
Resources: How State Sponsors Affect Insurgent Cohesion.''
\emph{International Studies Quarterly}, sqw033.
doi:\href{https://doi.org/10.1093/isq/sqw033}{10.1093/isq/sqw033}.

\hypertarget{ref-Tilly1992}{}
Tilly, Charles. 1992. \emph{Coercion, capital, and European states, AD
990-1992}. Malden, MA: Blackwell.

\hypertarget{ref-Warren2015}{}
Warren, T. Camber, and Kevin K. Troy. 2015. ``Explaining Violent
Intra-Ethnic Conflict : Group Fragmentation in the Shadow of State
Power.'' \emph{Journal of Conflict Resolution} 59 (3): 484--509.

\hypertarget{ref-Weinstein2007}{}
Weinstein, Jeremy M. 2007. \emph{Inside Rebellion}. Cambridge: Cambridge
University Press.

\hypertarget{ref-Wolford}{}
Wolford, Scott, David E. Cunningham, and William Reed. 2015. ``Why do
Some Civil Wars Have More Rebel Groups than Others? A Formal Model and
Empirical Analysis.'' \emph{Paper Presented at the 2015 Annual Meeting
of the International Studies Association, New Orleans, LA}.

\hypertarget{ref-Wood2003}{}
Wood, Elisabeth Jean. 2003. \emph{Insurgent Collective Action and Civil
War in El Salvador}. Cambridge: Cambridge University Press.

\hypertarget{ref-Wood2008a}{}
Wood, Reed M. 2008. ```A hand upon the throat of the nation': Economic
sanctions and state repression, 1976-2001.'' \emph{International Studies
Quarterly} 52 (3): 489--513.

\hypertarget{ref-Zeigler2016}{}
Zeigler, Sean M. 2016. ``Competitive alliances and civil war
recurrence.'' \emph{International Studies Quarterly} 60 (1): 24--37.


\end{document}
