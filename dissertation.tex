\documentclass[12pt,]{article}
\usepackage[]{tgpagella}
\usepackage{amssymb,amsmath}
\usepackage{ifxetex,ifluatex}
\usepackage{fixltx2e} % provides \textsubscript
\ifnum 0\ifxetex 1\fi\ifluatex 1\fi=0 % if pdftex
  \usepackage[T1]{fontenc}
  \usepackage[utf8]{inputenc}
\else % if luatex or xelatex
  \ifxetex
    \usepackage{mathspec}
  \else
    \usepackage{fontspec}
  \fi
  \defaultfontfeatures{Ligatures=TeX,Scale=MatchLowercase}
\fi
% use upquote if available, for straight quotes in verbatim environments
\IfFileExists{upquote.sty}{\usepackage{upquote}}{}
% use microtype if available
\IfFileExists{microtype.sty}{%
\usepackage{microtype}
\UseMicrotypeSet[protrusion]{basicmath} % disable protrusion for tt fonts
}{}
\usepackage[margin=1in]{geometry}
\usepackage{hyperref}
\hypersetup{unicode=true,
            pdftitle={Politics Among Rebels: The Causes of Division Among Dissidents},
            pdfauthor={David Bowden},
            pdfborder={0 0 0},
            breaklinks=true}
\urlstyle{same}  % don't use monospace font for urls
\usepackage{longtable,booktabs}
\usepackage{graphicx,grffile}
\makeatletter
\def\maxwidth{\ifdim\Gin@nat@width>\linewidth\linewidth\else\Gin@nat@width\fi}
\def\maxheight{\ifdim\Gin@nat@height>\textheight\textheight\else\Gin@nat@height\fi}
\makeatother
% Scale images if necessary, so that they will not overflow the page
% margins by default, and it is still possible to overwrite the defaults
% using explicit options in \includegraphics[width, height, ...]{}
\setkeys{Gin}{width=\maxwidth,height=\maxheight,keepaspectratio}
\IfFileExists{parskip.sty}{%
\usepackage{parskip}
}{% else
\setlength{\parindent}{0pt}
\setlength{\parskip}{6pt plus 2pt minus 1pt}
}
\setlength{\emergencystretch}{3em}  % prevent overfull lines
\providecommand{\tightlist}{%
  \setlength{\itemsep}{0pt}\setlength{\parskip}{0pt}}
\setcounter{secnumdepth}{5}
% Redefines (sub)paragraphs to behave more like sections
\ifx\paragraph\undefined\else
\let\oldparagraph\paragraph
\renewcommand{\paragraph}[1]{\oldparagraph{#1}\mbox{}}
\fi
\ifx\subparagraph\undefined\else
\let\oldsubparagraph\subparagraph
\renewcommand{\subparagraph}[1]{\oldsubparagraph{#1}\mbox{}}
\fi

%%% Use protect on footnotes to avoid problems with footnotes in titles
\let\rmarkdownfootnote\footnote%
\def\footnote{\protect\rmarkdownfootnote}

%%% Change title format to be more compact
\usepackage{titling}

% Create subtitle command for use in maketitle
\newcommand{\subtitle}[1]{
  \posttitle{
    \begin{center}\large#1\end{center}
    }
}

\setlength{\droptitle}{-2em}
  \title{Politics Among Rebels: The Causes of Division Among Dissidents}
  \pretitle{\vspace{\droptitle}\centering\huge}
  \posttitle{\par}
  \author{David Bowden}
  \preauthor{\centering\large\emph}
  \postauthor{\par}
  \predate{\centering\large\emph}
  \postdate{\par}
  \date{May 3, 2017}

\usepackage{setspace}


\usepackage{float}
\let\origtable\table
\let\endorigtable\endtable
\renewenvironment{table}[1][2] {
    \singlespacing
    \expandafter\origtable\expandafter[H]
} {
    \endorigtable
}

\begin{document}
\maketitle

{
\setcounter{tocdepth}{2}
\tableofcontents
}
\doublespacing

\section{Introduction}\label{introduction}

\subsection{Previous Work on the Structure of Rebel
Movements}\label{previous-work-on-the-structure-of-rebel-movements}

The existing literature and empirical record suggest that the number of
rebel groups active in a conflict is shaped by three broad processes.
The number of rebel groups can increase when existing groups splinter
into multiple factions. New groups can also emerge when previously
non-violent individuals mobilize and join the conflict. Finally, the
number of rebel groups can decrease when previously independent factions
form alliances. In the remainder of this chapter, I provide a definition
of each process, and review the existing explanations for each.

\subsubsection{Splintering}\label{splintering}

Existing rebel groups frequently splinter into multiple successor
organizations. In 1968, for example, a faction led by Ahmed Jibril broke
away from the Popular Front for the Liberation of Palestine (PFLP) to
form a new group, the Popular Front for the Liberation of
Palestine-General Command (PFLP-GC). While the two groups often
collaborated against Israel, they maintain distinct organizational
structures and membership bases, and operate in different areas. The
split was allegedly motivated by differing views of Marxist ideology and
military doctrine, with the PFLP pursuing a more extreme strategy of
attrition. Similar splits have occurred within dozens of rebel groups,
including the Communist Party of Burma, the Free Syrian Army and the
Sudan Liberation Army. In many cases the result is more than a nominal
separation. In Sri Lanka, for example, the Tamil Peoples Liberation
Tigers not only split from the Liberation Tigers of Tamil Eelam, but
also defected to the government side in the conflict (Staniland 2012).

A growing body of literature identifies several key determinants of
rebel group splintering. One subset of this research focuses on the role
of external actors, and particularly the government. For instance,
McLauchlin and Pearlman (2012) find that government repression provides
occasion for groups to evaluate their current leadership structure.
Pre-existing divisions within groups are likely to be exacerbated,
leading the group to move toward more factionalized leadership
structures. When group members are satisfied, however, conflict tends to
lead to even greater unity and centralization of authority. Whereas the
preceding studies essentially treat government repression as exogenous
to the internal politics of dissident groups, Bhavnani, Miodownik, and
Choi (2011) present evidence that governments deliberately stoke
tensions among their opponents, as they find that the Israeli government
increased conflict between Fatah and Hamas by undermining Hamas' control
of the Gaza and by tolerating Fatah's relationship with the Jordanian
military.

Another group of scholars emphasizes concerns about post-conflict
bargaining as the key determinant of dissident group cohesion. Christia
(2012) assumes that the winning coalition in a civil war receives
private benefits, which might include any rents available to the state,
and having some portion of its interests represented in the new
government. Thus, rebels have an incentive to form minimum winning
coalitions, so as to limit the number of coalition partners with whom
they must share benefits. Wolford, Cunningham, and Reed (2015) develop a
similar logic, theorizing that political factions have an interest in
joining conflicts so as to maximize the likelihood of their preferences
being represented in the post-war government, but the value of fighting
decreases as the number of parties with whom they expect to share power
increases. Yet, Christia (2012) suggests that this incentive to minimize
coalition size is moderated by the risk of being outside the winning
coalition, as there is a strong possibility of new waves of violence
between victorious rebels and rival rebel factions. She thus expects
coalitions to change frequently in response to battlefield events, with
factions bandwagoning with battle winners and shifting away from losing
coalitions. Findley and Rudloff (2012) similarly find fragmentation to
be most common among groups that have recently lost battles. This
implies that fragmentation is essentially a process of weak actors
becoming weaker.

A final category of explanations places the source of rebel group
cohesion in underlying social structure. Staniland (2014) argues that
insurgent organizations will be most stable when their central
leadership is able to exercise both vertical control over its
rank-and-file members, and horizontal control over its constituent
groups. This is most likely to occur when insurgencies draw from
existing organizations with extant social ties of this sort, which might
include former anti-colonial movements or ethnic political parties.
Organizations are likely to fragment when constituent groups have a high
degree of autonomy or control over individual members is limited
(Staniland 2014, Ch. 2-3). Asal, Brown, and Dalton (2012) emphasize
similar factors, arguing that organizations with factionalized
leadership structures are at risk of fragmentation, while groups with
more consolidated power structures will tend to remain cohesive.
Finally, Warren and Troy (2015) suggest that group size plays an
important role, as small groups are able to police themselves and
resolve conflicts, whereas larger groups are more likely to experience
infighting.

Each of these studies makes an important contribution to our
understanding of conflict complexity, and sheds light on the broader
interests and organizational challenges present in rebel movements. Yet,
while the fragmentation of existing groups accounts for a substantial
portion of multi-rebel conflicts, other processes are at work in the
majority of cases. Indeed, only 26.6\%\footnote{These figures are
  calculated using data on conflict participation from Pettersson and
  Wallensteen (2015) and actor attributes from (Uppsala Conflict Data
  Program 2015). I code a conflict episode as a separate war if it
  occurs following at least two calendar years of inactivity.
  Secessionist movements are treated as separate conflicts from bids to
  overthrow the central government, and separate from each other if they
  concern different territories.} of the rebel groups that join ongoing
civil wars splintered from an existing group, and only 9.7\% are
agglomerations of existing groups. Thus, nearly two-thirds of the groups
that join conflicts\footnote{The pattern is even more stark if one looks
  at conflict-years, as over 95\% of conflict years with multiple
  government-rebel dyads include at least one rebel group that is
  neither a splinter organization nor the source of one.} (only 20\% of
multi-dyadic conflicts have multiple rebel groups from the outset) do
not appear to be the product of existing combatants reconfiguring, but
rather are the result of an entirely new group of combatants entering
the fray. I propose an integrated approach that accounts for both the
fragmentation of existing groups, and the entry of new groups to the
conflict.

\subsubsection{Alliance Formation}\label{alliance-formation}

\subsubsection{Mobilization}\label{mobilization}

Few, if any, studies directly consider the phenomenon of new rebel
groups joining ongoing conflicts. The literature on contagion is perhaps
most relevant. \citet{Gleditsch2007} finds that transnational ethnic
groups and political and economic linkages between states can provide
channels for civil war to spread across international boundaries. Other
scholars find that secessionist \citep{Ayres2000} and ethnic
\citep{Lane2016} conflict often spread through processes of contagion,
with the rebellion of one group seemingly inspiring those in neighboring
areas to take up arms themselves. Such transnational processes might
shape opportunities for multiple rebellions to emerge by increasing the
availability of weapons, spreading tactical knowledge, or diverting
government attention to foreign conflicts. Similarly, transnational
motives for conflict may come in the form of grievances becoming clearer
and more salient in light of events in neighboring countries, as
happened during the Arab Spring, or the expected probability of a
successful rebellion shifting upward in response to nearby events. Yet,
rebel groups that are themselves transnational, operating in multiple
countries \citep[see][]{salehyan07} account for only 10.8\% of conflict
joiners.

\section{Alliance Formation}\label{alliance-formation-1}

\subsection{Introduction}\label{introduction-1}

Theories of civil war tend to focus on individual- or group-level
motives (e.g. Gurr 1970; Collier and Hoeffler 2004) or opportunities
(e.g. Fearon and Laitin 2003) for rebellion, while giving little
attention to the organization of dissent into rebel groups and
coalitions. Even those studies which do explicitly consider rebel group
formation tend to focus on group attributes such as discipline (e.g.
Weinstein 2007), and do not consider the possibility that rebels do not
always form a single group. Yet, 44\% of civil conflicts feature at
least two rebel groups challenging the government.\footnote{Source:
  Pettersson and Wallensteen (2015).} Over the course of the Chadian
Civil War, for instance, 25 distinct rebel groups fought against the
government. Conflicts in Afghanistan in the 1980's, Somalia in the
1990's, and Sudan in the 2000's have been similarly complex. The ongoing
civil war in Syria is contested by at least two dozen armed groups. Even
ethnically-homogeneous, geographically-concentrated movements with
common goals, such as the Karen secessionist campaign in Myanmar, often
fragment into multiple rebel groups. Furthermore, the number of groups
operating in these conflicts often varies greatly over time. The
existing literature offers many useful insights to the conditions under
which civil war will emerge, but it has few explanations of the
structure of rebel movements.

While little attention has been given to the sources of rebel movement
structure, several studies suggest that such configurations can have
deleterious consequences. Conflicts with multiple rebel groups last
longer than dyadic competitions (Cunningham 2006; Cunningham, Gleditsch,
and Salehyan 2009; Akcinaroglu 2012). Furthermore, Cunningham,
Gleditsch, and Salehyan (2009) find that the presence of multiple
government-rebel dyads decreases the likelihood of peace agreements and
increases the likelihood of rebel victories, though Findley and Rudloff
(2012), find that fragmented rebel movements are often associated with
an \emph{increased} likelihood of negotiated settlement. Relatedly,
Atlas and Licklider (1999) and Zeigler (2016) find that episodes of
conflict renewal often occur between formerly allied rebel factions.
Finally, conflicts with multiple dyads feature more fatalities than
dyadic ones.\footnote{Source: my own analysis using data from Sundberg
  (2008).} Clearly, conflicts with multiple rebel groups comprise one of
the most severe subsets of civil wars. Thus, understanding the causes of
multi-dyadic conflict is of great normative and policy importance.

I seek to address this gap by explaining one of the primary determinants
of rebel movement structure --- the formation of alliances between rebel
factions. Which rebel groups are likely to form alliances? With whom are
they likely to ally? While alliances cannot account for all of the
variation in the number of rebel groups in a conflict --- the
fragmentation of existing groups and the entry of previously non-violent
groups to the conflict are also important processes --- alliance ties
tend to predict deeper integration between rebel groups. Many rebel
alliances evolve into umbrella organizations with shared command, and
weak rebel groups are frequently absorbed by alliance partners. Thus,
alliance formation is a crucial determinant of whether conflicts become
less complex over time.

First and foremost, this work advances our understanding of the
complexity of civil conflict in terms of the number and arrangement of
actors. This builds on the growing literature on another facet of
complexity --- the fragmentation of existing groups (see Cunningham,
Gleditsch, and Salehyan 2009; Pearlman and Cunningham 2011; Staniland
2014) --- and presents contrasting evidence to the few existing studies
of rebel alliances (Christia 2012; Bapat and Bond 2012), which focus on
relative capability as the key driver of coalition building.
Furthermore, examining the relationships between rebel groups sheds new
light on debates about the motives behind rebellion (e.g. Collier and
Hoeffler 2004). For instance, if rebellion is fundamentally about ethnic
or religious grievances, we might expect to see the emergence of
coalitions with homogeneous identities. If, by contrast, rebels are
motivated by the desire for profits from natural resources or illicit
activities, we might see groups with access to such revenues seek to
limit the number of combatants with whom they share their spoils,
irrespective of common identity.

I proceed with a review of the literature on relations between rebel
groups. Next, I outline the potential benefits rebels might receive by
forming alliances. Subsequently, I explore the conditions under which
rebels will elect to engage in such cooperation. Finally, I present
results from an inferential network model applied to the Syrian Civil
War.

\subsection{Relations Among Rebels}\label{relations-among-rebels}

Relations among rebel groups remains one of the most underexplored
aspects of civil war. The vast majority of existing studies focus on
conflict between non-state actors. A few studies have, however, examined
conflict between rebel groups. In the most comprehensive study to date,
Fjelde and Nilsson (2012) suggest that rebels fight each other for
control over resources such drug supplies or valuable terrain. Greater
resource endowments should lead to better postwar bargains with the
government in the long run, and greater ability to sustain a rebellion
in the short run. The authors find that this logic is most likely to
prevail in the presence of natural resources, territories that are not
controlled by the government, militarily weak governments, and
significant power asymmetries among rebels. Atlas and Licklider (1999)
and Zeigler (2016) find that this dynamic can also arise in the
aftermath of conflicts, as the main fighting in renewed civil wars is
often between previously allied rebel groups.

Another strand of literature examines the emergence of conflict within
previously coherent movements. Bakke, Cunningham, and Seymour (2012)
conceptualize the fragmentation of rebels movements as varying in terms
of the raw number of organizations, the degree of institutionalization
unifying the organizations, and the distribution of power among them.
They expect a greater general likelihood of infighting in more
fragmented movements, particularly those with large numbers of groups
and low degrees of institutionalization. Asal, Brown, and Dalton (2012)
find evidence that largely supports this claim, showing that
ethnopolitical movements with factionalized leadership structures are
most likely to experience splits. Similarly, Cunningham, Bakke, and
Seymour (2012) find that infighting is most likely in self-determination
movements with large numbers of factions, and that the emergence of new
factions is especially likely to trigger violence. Staniland (2014)
similarly emphasizes pre-existing social structure, arguing that the
probability of an insurgent group splintering is shaped by the strength
of social ties in the organization from which it emerged. Others see
infighting as contingent on the conflict process. Christia (2012) finds
that rebel groups and coalitions tend to fragment when battlefield
losses exacerbate divisions between faction leaders, while McLauchlin
and Pearlman (2012) find that repression can deepen rifts within
movements that are already divided, but cohesive movements may become
further solidified by repression.

There is also a substantial literature on intra-ethnic violence. Lilja
and Hultman (2011) find that that Tamil rebels used violence against
co-ethnic civilians to control populations and the resources they hold,
and against coethnic armed groups to establish dominance within the
ethnic group. Staniland (2012) finds that these patterns intra-ethnic
violence tend to be self-reinforcing, as violence within ethnic
insurgencies is the primary cause of the defection of some subsets of
ethnic groups to the opposing side in the conflict. (Warren and Troy
2015) seek to explain which ethnic groups are likely to experience such
fragmentation, finding a curvilinear relationship between the size of an
ethnic group and its probably of experiencing infighting. Small ethnic
groups have the ability to police themselves, limiting violence, and
intra-ethnic violence in large groups is likely to be met with
government intervention. Thus, only moderate-sized groups tend to
experience internal violence.

Finally, a few studies consider the formation of alliances among various
types of militant organizations. Asal and Rethemeyer (2008) and Horowitz
and Potter (2013) conduct network analyses of alliance formation among
terrorist groups, arguing that such arrangement are used to aggregate
capabilities and share tactics. Bapat and Bond (2012) model the logic of
alliance formation among rebel groups. They assume that alliances carry
two significant costs: the dilution of each constituent group's agenda,
and the risk of having one's private information sold to the government
by an ally. Consistent with this theory, they find alliances to be most
common when an outside state can enforce agreements, and when all rebel
groups involved are strong enough to avoid the temptation of defecting
to the government side Christia (2012) similarly emphasizes capability,
arguing that neorealist balancing theory from international relations
explains alignments in civil wars. When one coalition - a group of
rebels or government-aligned forces - becomes too powerful, other groups
will band together to prevent their own destruction. But similar Bapat
and Bond (2012), Christia (2012) argues that this mechanism is
constrained by a desire to maximize one's share of the post-war spoils.
Thus, rebels realign frequently, seeking to form minimum winning
coalitions. While shared identity appears on the surface to be an
important determinant of rebel alignments, Christia views these
narratives as post-hoc justifications aimed at legitimizing decisions
that are really driven mostly by power. Some important aspects of
alliance formation are beyond the scope of the existing studies,
however. Namely, while relative power considerations can potentially
account for why rebel groups form alliances, and when they will alter
their ties, it does not explain why groups choose a particular partner
when multiple options are available. Christia suggests that these
decisions are shaped by personal relationships between rebel elites, but
does not give this question extended consideration in her empirical
analysis. Horowitz and Potter (2013) find that militants prefer to ally
with powerful groups, but their focus is largely on transnational
networks of terrorists and insurgents, rather than alliance formation
within a particular conflict. I seek to resolve this gap by explaining
not only whether, but also with whom rebel groups will choose to form
alliances.

\subsection{A Theory of Rebel Alliance
Formation}\label{a-theory-of-rebel-alliance-formation}

\subsubsection{Rebel Factions and Their
Interests}\label{rebel-factions-and-their-interests}

I start from the assumption that governments in civil wars are opposed
by one or more dissident factions. I define a faction as a set of people
with relatively homogeneous beliefs and identities, that is capable of
acting as a group. For example, a communist party might have Leninist
and Maoist factions, and an ethnonationalist movement might have
Christian and Muslim factions. Faction members likely will not share
identical beliefs on every political question, and may not have
identical racial, ethnic, or religious backgrounds. This might be
especially true of factions organized around loyalty to a specific
individual or locale. A faction will, however, tend to be unified on the
most salient political and identity issues of the moment.

In some contexts, the faction may not be an appropriate unit of
analysis. For example, rebel groups often do engage in recruiting new
members on an individual basis, and an individual-level approach would
be appropriate for studying such a phenomenon. For the formation and
restructuring of rebel groups, however, decisions tend to be made by
collectives. Staniland (2014) argues that rebel groups tend to emerge
from existing organizations, rather than spontaneous collections of
individuals. My own data collection shows similar patterns, with 97\% of
rebel groups since 1946 having origins in a pre-existing organization.
When rebel groups fragment, abandonment of the existing group is often
done by entire sub-units. For example Staniland (2014) finds that
fragmentation often takes the form of local brigades breaking away from
a central organization, as was the case for with al-Qaeda in Iraq, which
saw many local Sunni militias defect and begin cooperating with the US.
Similarly, Christia (2012) sees realignment as being driven by mid-level
rebel commanders, who generally take a cadre of loyal forces with them
when initiating and breaking alliances.

Factions can structured in a variety of arrangements during a conflict.
A faction can resort to violence on an individual basis, resulting in a
rebel group with mostly homogeneous preferences and identities. In other
cases a faction may ally with others to former a larger, but more
heterogeneous rebel group. Finally, factions may remain non-violent,
using peaceful tactics to oppose the government. These arrangement are
dynamic, however. Previously independent factions may form alliances,
previusly aligned factions may choose to break alliances, previously
non-violent factions may choose to enter a conflict, and previously
violent factions may choose to demobilize. I argue that two broad
concerns shape each these decisions.

First, I assume that factions have genuine political interests. This
assumption is not necessarily obvious in light of the long-running
debate as to whether rebellion is motivated by public or private
concerns (see Collier and Hoeffler 2004). Indeed, some scholars have
gone as far as to posit that rebellion is little more than glorified
criminal activity aimed at controlling natural resources and illicit
trades (Mueller 2000), or that individual participation is often
motivated by a desire to settle personal disputes (Kalyvas 2006). I do
not reject the notion that rebels value such things (indeed, see the
second assumption below); I simply contend that greed and grievance are
a false dichotomy. Most rebel groups articulate a political platform of
some variety. This might take the form of a comprehensive ideological
program such as a communist revolution, or a more narrow concern such as
land reform or self-determination for a particular ethnic group. While
earlier work found greater support for the greed hypothesis, suggesting
that such political rhetoric is merely a veneer on more selfish motives,
recent studies using higher-quality data have found that political
grievances, and particularly ethnic discrimination, to be among the
strongest predictors of civil war (Cederman, Wimmer, and Min 2010).

Second, however, I expect that factions will seek to maximize their
individual power. Even if rebels are primarily motivated by political
interests, material resources are an important means to achieving such
goals. A faction's influence in any postwar order is likely to be shaped
in large part by its power. If a faction retains enough fighting
capability to re-open violence, other actors interested in peace will
have an interest in accommodating many of their demands. Relatedly, a
faction will have difficulty trusting any concessions it wins from more
powerful actors. Thus, retaining capability at the end of a conflict is
likely to be advantageous with respect to advancing one's political
goals (Nygård and Weintraub 2015). Furthermore, greater power allows a
faction greater autonomy from fellow rebels. Political power is a finite
good, and as a group's capabilities increase, the number of other rebels
with whom they must share it decreases (Christia 2012; Bapat and Bond
2012). Finally, even if it is not their primary motive, rebels do use
material resources to incentivize recruitment and retention efforts
(Weinstein 2007), and to enrich top leadership.

\subsubsection{The Value of Rebel
Alliances}\label{the-value-of-rebel-alliances}

Frequently, rebel factions engage in military collaboration with other
non-state actors. This can range from an agreement not to target each
other, to a divisions of territory, to joint campaigns on the
battlefield, to full mergers. These alliances can be valuable for a
number of reasons. First, alliances aggregate capabilities. This is
perhaps the most common conception of alliances in international
politics (see Bennett 1997), and it has been proposed as a motive for
rebel alliances as well {[}Bapat and Bond (2012); Horowitz and Potter
(2013). The logic of capability aggregation differs somewhat between
international and civil conflicts. Whereas international alliances
aggregate capabilities by bringing states into a conflict in which they
might not otherwise participate, rebel groups by definition are already
participating in conflict. Nevertheless, these alliances can bring great
value because rather than simply aggregating, they can concentrate
capabilities in space and time. For example, two rebel groups might be
unable to capture a government-held town on their own, but in a joint
operation would be sufficiently powerful to do so.

Second, alliances can allow for burden-sharing and specialization.
Burden-sharing has been offered as an explanation for international
alliances such as NATO (Sandler and Forbes 1980), though it may not
occur under all circumstances (see Olson and Zeckhauser 1966). Alliances
can ensure that a single rebel group is not responsible for defeating
the government, and might serve as a mechanism for reigning in the
temptation to free ride off of another group's efforts. Relatedly,
alliances can facilitate specialization by rebel groups. For instance,
one alliance partner might specialize in holding territory, while
another specializes in launching offensives in new areas. Furthermore,
they can share strategies and technical information. For example, Hamas
is believed to have learned how to use suicide bombings through its
alliance with Hezbollah (Horowitz and Potter 2013).

Third, alliances can manage conflict between members and ensure that
their resources are directed toward common enemies. Weitsman (1997)
argues that alliances often serve to tether powerful states to one
another, so as to reduce the probability of conflict between them.
Gibler (1996) finds that alliance treaties are often used to settle
territorial disputes between the signatories. Similar alliances can be
seen in civil wars, for example as a number of Syrian rebel groups
agreed to focus their efforts in different regions of the country. This
allows rebels to avoid conflict with each other. Compliance with such
agreements is incentivized by the fact that reneging on the territorial
arrangement would likely result in the loss of the other benefits of the
alliance, such as capability aggregation.

Fourth, operating as an alliance bloc may be beneficial to the members
groups in bargaining situations. An alliance with a set of coordinated
demands might command greater bargaining leverage than individual
members, who collectively have similar power, but a more disparate set
of demands. Perhaps more crucially, alliances might mitigate credible
commitment problems. Peaceful settlements to conflicts can be derailed
by concerns that the other side will not adhere to the agreement (Fearon
1995). In civil wars, this is often borne out by extreme ``spoiler''
factions. A rebel commitment to a peace agreement is more likely to be
viewed as credible if it has formal control over other factions.

\subsubsection{The Costs of Alliances}\label{the-costs-of-alliances}

While the benefits are often many, most alliances between rebel groups
are not without cost. The post-war political outcome, whether it comes
in the form of a rebel victory or a compromise with the incumbent
government, is likely to be shaped by all factions within the winning
coalition. Thus, allying with another group holding differing ideologies
and interests will tend to force a rebel faction to compromise on at
least some issues, or to de-emphasize certain priorities. If, as I
assume, rebels are motivated by political goals, the value of an
alliance will decrease as its ideological similarity to its alliance
partners decreases (Bapat and Bond 2012). Furthermore, any private
benefits deriving from the conflict outcome (such as seats in a post-war
legistlature) must be divided among the members of the winning alliance
(Christia 2012). These concerns should tend to constrain the value of
alliances in civil war. The existing literature finds that these
concerns limit the size of rebel coalitions (Christia 2012). Logically,
they should also shape the choice of partners with whom rebels ally.

\subsubsection{The Choice of Alliance
Partners}\label{the-choice-of-alliance-partners}

I expect that the decision to form an alliance with a particular group
is shaped by two broad considerations. The first is the ideological
similarity of the two groups. The second is the potential gain in
capability. Consistent with the existing literature, I view the current
material capabilities of a potential ally as a crucial factor, with more
powerful groups making more attractive alliance partners. I depart from
the literature (e.g. Christia 2012), however, by also considering the
importance of access to future sources of power. Specifically, I expect
that a group will evaluate a potential alliance partner not only on its
current level of capability, but also on the extent to which the group
is a rival for access to future sources of power, such as natural
resources or civilian populations. In other words, a rebel group with
enough power to normally be an attractive partner may not be if it's
strength is drawn from similar support bases as one's own group. By
contrast, a relatively weak group with a completely non-overlapping
support base might be an attractive ally.

A rebel group's support base is shaped by a mix of external factors such
as the presence of natural resources and foreign sponsors, as well as
its objectives. Some ideological objectives provide rebel groups with
somewhat malleable support bases, such as those that entail the
provision of public, non-rival goods to society. It is comparatively
easy for groups of this sort to minimize the overlap between their
support bases. By contrast, rebels that pursue private, rival goods or
interests specific to certain societal groups are likely to be in
competition with rebels advancing similar objectives. In the remainder
of this section I classify various rebel objectives on this dimension.

Most non-sectarian ideological interests should fall into the category
of public, non-rival goods. If two groups each prefer a similar goal,
such as a redistributive welfare system, a greater role for Islam in
government, or a devolution of power to regional governments, they will
be able to enjoy the benefits of such policies regardless of which group
enacts them. All else equal, goals of this sort should create common
interests among the rebels who share them. Furthermore, policies of this
sort tend not to have pre-defined constituencies. A rebel group based on
ideology could potential convert new members or civilian supporters to
its cause by spreading its beliefs. As ideologies of this sort are
generally not tied to a specific ethnicity, religion, or geographic
area, the pool of potential converts is quite large. Thus, groups
centered around ideologies of this sort should have high potential for
cooperation, as they are relatively unlikely to be rivals for support.
The value of cooperation will be especially high for groups that have
similar non-sectarian ideologies.

\emph{H1: Rebel groups with similar non-sectarian ideologies should be
more likely to form alliances than other rebel dyads}

While groups with similar non-sectarian interests should tend not to
come into competition until late in conflicts, for groups representing
identity-based interests, the effect is contingent on the size of the
group and malleability of group boundaries. The reason for this lies in
the fact that many rebel groups rely on civilian populations for
material support (Weinstein 2007), and the types of goals a group
pursues is an important determinant of the malleability of civilian
support coalitions. A group with broad-based policy goals might be able
to persuade or coerce almost any group of citizens to support it. Thus,
until a very large portion of the civilian population has been captured,
groups sharing these types of goals will not be in competition over
support as they can simply carve out different coalitions. Similar
dynamics should occur among groups pursuing the interests of large or
social groups, such as the majority ethnic or religious group. For
example two Syrian rebel groups seeking to replace the Alawite-dominated
Assad regime with one that embraces Sunni doctrine should find that
civilian support is not particularly scarce given that Syria is
majority-Sunni. Similarly, groups advocating the interests of social
groups with fluid boundaries should tend to have opportunities to
capture new civilian support rather than competing with similar groups
over existing support. For instance, for a group advocating a
Salafi-Jihadi ideology, any Sunni Muslim might serve as a potential
convert.

Rebels representing minority social groups, however, should tend to come
into conflict more quickly. Groups of this sort must draw their support
from a social base that is both smaller and more likely to be tapped out
than the bases of more broadly defined groups, and that is more rigidly
bounded. A rebel group aimed at advancing the interests of a particular
ethnic or religious group is unlikely to attract support from non-group
members. Even if it was able to do so, this might its standing with
co-ethnics/co-religionists, as rival groups could claim that it is
watering down its agenda. In other words, socially-defined rebel groups
seeking to expand the pool of potential support might be vulnerable to
outbidding appeals. In short, I expect that groups with the agenda of
advancing the interests of majority ethnic or religious groups will be
likely to cooperate with groups holding similar interests. Groups
representing social minorities, however, should be unlikely to
cooperate.\footnote{In the present analysis, however, the Kurds are the
  only group to whom this logic is likely to apply, and thus I do not
  test this hypothesis as it would essentially be a dummy variable for
  the one Kurdish-Kurdish dyad.}

\emph{H2: Rebel groups representing the same majority ethnic or
religious groups will be more likely to form alliances than other dyads}

Groups seeking to control the same territory should face a similar
problem of rival consumption. Because secessionist claims tend to have
well-defined geographic and/or ethnic boundaries, rebel groups
representing such claims are likely to be in competition over a fixed
pool of support. Thus, I expect that groups making similar territorial
claims will be unlikely to cooperate.

\emph{H3: Rebel groups with overlapping territorial claims will be less
likely to form alliances than other dyads}

```

\subsection{Conclusion}\label{conclusion}

I have argued that contrary to some theoretical treatments, rebel groups
do care about political aims. This fact should lead alliance ties to be
most common among groups sharing similar goals. Indeed, I find that in
the Syrian Civil War, shared political goals are the single most
important determinant of alliance partners. I do not find support for
the notion that more powerful groups should be more likely to form
alliances. While I do not find evidence of religious homophily, that may
be an artifact of the limited diversity of Syrian rebels. Finally, I
find a null relationship between shared territorial ambitions and
alliances, where I expected a negative relationship.

The finding on the importance of political goals contrasts with multiple
existing theories of rebellion. The importance of political goals
contrasts with the greed model of civil war, which views rebellion as
being primarily aimed at procuring private material benefits for
members. It also calls into question purely power-based accounts
(Christia 2012), which expect rebels to be concerned with little else
but winning. In addition, these results can help us to predict the
dynamics of civil conflicts. If we observe a conflict with many rebel
factions, but these groups share similar goals, we might expect the
movement to aggregate over time. If these groups have disparate
interests, however, there is a strong possibility that the conflict will
remain highly fragmented. Given the severity associated with more
complex conflicts, the ability to make predictions of this sort is
highly valuable.

Further research in this area is needed. My hypothesized effects may
simply be conditional on factors that I have yet to account for. For
example, access to material support from an outside actor should reduce
competition over civilian support bases or territory. In addition, this
work should be replicated in other cases. As one of the most complex
civil wars on record, it is possible that the dynamics in Syria do not
apply in other conflicts. Finally, future work should move beyond
explaining the formation of networks and explore the effects of network
structure on rebel behavior. For instance, are more densely networked
rebel coalitions more resilient to anti-insurgent campaigns? Do certain
tactics diffuse across rebel networks?

\section{References}\label{references}

\indent

\setlength{\parindent}{-0.2in} \setlength{\leftskip}{0.2in}
\setlength{\parskip}{8pt}

\singlespacing

\hypertarget{refs}{}
\hypertarget{ref-Akcinaroglu2012a}{}
Akcinaroglu, Seden. 2012. ``Rebel Interdependencies and Civil War
Outcomes.'' \emph{Journal of Conflict Resolution} 56 (5): 879--903.

\hypertarget{ref-Asal2008}{}
Asal, Victor, and R. Karl Rethemeyer. 2008. ``The Nature of the Beast:
Organizational Structures and the Lethality of Terrorist Attacks.''
\emph{The Journal of Politics} 70 (2): 437--49.

\hypertarget{ref-Asal2012}{}
Asal, Victor, Mitchell Brown, and Angela Dalton. 2012. ``Why Split?
Organizational Splits among Ethnopolitical Organizations in the Middle
East.'' \emph{Journal of Conflict Resolution} 56 (1): 94--117.

\hypertarget{ref-Atlas1999}{}
Atlas, Pierre M., and Roy Licklider. 1999. ``Conflict Among Former
Allies After Civil War Settlement : Sudan , Zimbabwe , Chad , and
Lebanon.'' \emph{Journal of Peace Research} 36 (1): 35--54.

\hypertarget{ref-Bakke2012a}{}
Bakke, Kristin M., Kathleen Gallagher Cunningham, and Lee J. M. Seymour.
2012. ``A Plague of Initials: Fragmentation, Cohesion, and Infighting in
Civil Wars.'' \emph{Perspectives on Politics} 10 (2): 265--83.

\hypertarget{ref-Bapat2012}{}
Bapat, Navin, and Kanisha Bond. 2012. ``Alliances between Militant
Groups.'' \emph{British Journal of Political Science} 42 (4): 793--824.

\hypertarget{ref-Bennett1997}{}
Bennett, D. Scott. 1997. ``Testing Alternative Models of Alliance
Duration, 1816-1984.'' \emph{American Journal of Political Science} 41
(3): 846.

\hypertarget{ref-Bhavnani2011}{}
Bhavnani, Ravi, Dan Miodownik, and Hyun Jin Choi. 2011. ``Three Two
Tango: Territorial Control and Selective Violence in Israel, the West
Bank, and Gaza.'' \emph{Journal of Conflict Resolution} 55 (1): 133--58.
doi:\href{https://doi.org/10.1177/0022002710383663}{10.1177/0022002710383663}.

\hypertarget{ref-Cederman2010}{}
Cederman, Lars-Erik, Andreas Wimmer, and Brian Min. 2010. ``Why do
ethnic groups rebel?: New data and analysis.'' \emph{World Politics} 62
(1): 87--98.

\hypertarget{ref-Christia2012}{}
Christia, Fotini. 2012. \emph{Alliance Formation in Civil Wars}.
Cambridge: Cambridge University Press.

\hypertarget{ref-Collier2004}{}
Collier, Paul, and Anke Hoeffler. 2004. ``Greed and grievance in civil
war.'' \emph{Oxford Economic Papers} 56 (4): 563--95.

\hypertarget{ref-Cunningham2006}{}
Cunningham, David E. 2006. ``Veto Players and Civil War Duration.''
\emph{American Journal of Political Science} 50 (4): 875--92.

\hypertarget{ref-Cunningham2009}{}
Cunningham, David E., Kristian Skrede Gleditsch, and Idean Salehyan.
2009. ``It Takes Two: A Dyadic Analysis of Civil War Duration and
Outcome.'' \emph{Journal of Conflict Resolution} 53 (4): 570--97.

\hypertarget{ref-Cunningham2012a}{}
Cunningham, Kathleen Gallagher, Kristin M. Bakke, and Lee J. M. Seymour.
2012. ``Shirts Today , Skins Tomorrow : Dual Contests and the Effects of
Fragmentation in Self-Determination Disputes.'' \emph{Journal of
Conflict Resolution} 56 (1): 67--93.

\hypertarget{ref-fearon95}{}
Fearon, James D. 1995. ``Rationalist Explanations for War.''
\emph{International Organization} 49 (3): 379--414.

\hypertarget{ref-fearonlaitin03}{}
Fearon, James D., and David D. Laitin. 2003. ``Ethnicity, Insurgency,
and Civil War.'' \emph{American Political Science Review} 97 (1):
75--90.

\hypertarget{ref-Findley2012}{}
Findley, Michael, and Peter Rudloff. 2012. ``Combatant Fragmentation and
the Dynamics of Civil Wars.'' \emph{British Journal of Political
Science} 42 (4): 879--901.

\hypertarget{ref-Fjelde2012}{}
Fjelde, Hanne, and Desiree Nilsson. 2012. ``Rebels against Rebels:
Explaining Violence between Rebel Groups.'' \emph{Journal of Conflict
Resolution} 56 (4): 604--28.

\hypertarget{ref-Gibler1996}{}
Gibler, Douglas M. 1996. ``Alliances That Never Balance: The Territorial
Settlement Treaty.'' \emph{Conflict Management and Peace Science} 15
(1): 75--97.

\hypertarget{ref-gurr70}{}
Gurr, Ted Robert. 1970. \emph{Why Men Rebel}. Princeton, NJ: Princeton
University Press.

\hypertarget{ref-Horowitz2013}{}
Horowitz, Michael C., and Philip B. K. Potter. 2013. ``Allying to Kill:
Terrorist Intergroup Cooperation and the Consequences for Lethality.''
\emph{Journal of Conflict Resolution} 58 (2): 199--225.

\hypertarget{ref-Kalyvas2006}{}
Kalyvas, Stathis N. 2006. \emph{The Logic of Violence in Civil War}.
Cambridge: Cambridge University Press.

\hypertarget{ref-Lilja2011}{}
Lilja, Jannie, and Lisa Hultman. 2011. ``Intraethnic Dominance and
Control: Violence Against Co-Ethnics in the Early Sri Lankan Civil
War.'' \emph{Security Studies} 20 (2): 171--97.

\hypertarget{ref-McLauchlin2012}{}
McLauchlin, Theodore, and Wendy Pearlman. 2012. ``Out-Group Conflict,
In-Group Unity?: Exploring the Effect of Repression on Intramovement
Cooperation.'' \emph{Journal of Conflict Resolution} 56 (1): 41--66.

\hypertarget{ref-mueller00}{}
Mueller, John. 2000. ``The Banality of Ethnic War.'' \emph{International
Security} 25 (1): 42--70.

\hypertarget{ref-Nygard2014}{}
Nygård, Håvard Mokleiv, and Michael Weintraub. 2015. ``Bargaining
Between Rebel Groups and the Outside Option of Violence.''
\emph{Terrorism and Political Violence} 27 (3): 557--80.

\hypertarget{ref-Olson1966}{}
Olson, Mancur, and Richard Zeckhauser. 1966. ``An economic theory of
alliances.'' \emph{The Review of Economics and Statistics} 48 (3):
266--79.

\hypertarget{ref-Pearlman2011a}{}
Pearlman, Wendy, and Kathleen Gallagher Cunningham. 2011. ``Nonstate
Actors, Fragmentation, and Conflict Processes.'' \emph{Journal of
Conflict Resolution} 56 (1): 3--15.

\hypertarget{ref-Pettersson2015a}{}
Pettersson, Therése, and Peter Wallensteen. 2015. ``Armed conflicts,
1946-2014.'' \emph{Journal of Peace Research} 52 (4): 536--50.

\hypertarget{ref-Sandler1980}{}
Sandler, Todd, and John F. Forbes. 1980. ``Burden Sharing, Strategy, and
the Design of NATO.'' \emph{Economic Inquiry} 18 (3): 425--44.

\hypertarget{ref-Staniland2012d}{}
Staniland, Paul. 2012. ``Between a Rock and a Hard Place: Insurgent
Fratricide, Ethnic Defection, and the Rise of Pro-State
Paramilitaries.'' \emph{Journal of Conflict Resolution} 56 (1): 16--40.

\hypertarget{ref-Staniland2014}{}
---------. 2014. \emph{Networks of Rebellion: Explaining Insurgent
Cohesion and Collapse}. Ithaca, NY: Cornell University Press.

\hypertarget{ref-Sundberg2008a}{}
Sundberg, Ralph. 2008. ``Collective Violence 2002-2007: Global and
Regional Trends.'' In \emph{States in Armed Conflict 2007}, edited by
Lotta Harbom and Ralph Sundberg. Uppsala: Universitetstryckeriet.

\hypertarget{ref-ucdpactor}{}
Uppsala Conflict Data Program. 2015. ``UCDP Actor Dataset 2.2-2015.''
Uppsala University.

\hypertarget{ref-Warren2015}{}
Warren, T. Camber, and Kevin K. Troy. 2015. ``Explaining Violent
Intra-Ethnic Conflict : Group Fragmentation in the Shadow of State
Power.'' \emph{Journal of Conflict Resolution} 59 (3): 484--509.

\hypertarget{ref-Weinstein2007}{}
Weinstein, Jeremy M. 2007. \emph{Inside Rebellion}. Cambridge: Cambridge
University Press.

\hypertarget{ref-Weitsman1997}{}
Weitsman, Patricia. 1997. ``Intimate Enemies: The Politics of Peacetime
Alliances.'' \emph{Security Studies} 7 (1): 156--93.

\hypertarget{ref-Wolford}{}
Wolford, Scott, David E. Cunningham, and William Reed. 2015. ``Why do
Some Civil Wars Have More Rebel Groups than Others? A Formal Model and
Empirical Analysis.'' \emph{Paper Presented at the 2015 Annual Meeting
of the International Studies Association, New Orleans, LA}.

\hypertarget{ref-Zeigler2016}{}
Zeigler, Sean M. 2016. ``Competitive alliances and civil war
recurrence.'' \emph{International Studies Quarterly} 60 (1): 24--37.


\end{document}
