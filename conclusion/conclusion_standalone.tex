\documentclass[12pt,]{book}
\usepackage[]{tgpagella}
\usepackage{amssymb,amsmath}
\usepackage{ifxetex,ifluatex}
\usepackage{fixltx2e} % provides \textsubscript
\ifnum 0\ifxetex 1\fi\ifluatex 1\fi=0 % if pdftex
  \usepackage[T1]{fontenc}
  \usepackage[utf8]{inputenc}
\else % if luatex or xelatex
  \ifxetex
    \usepackage{mathspec}
  \else
    \usepackage{fontspec}
  \fi
  \defaultfontfeatures{Ligatures=TeX,Scale=MatchLowercase}
\fi
% use upquote if available, for straight quotes in verbatim environments
\IfFileExists{upquote.sty}{\usepackage{upquote}}{}
% use microtype if available
\IfFileExists{microtype.sty}{%
\usepackage{microtype}
\UseMicrotypeSet[protrusion]{basicmath} % disable protrusion for tt fonts
}{}
\usepackage[margin=1in]{geometry}
\usepackage{hyperref}
\hypersetup{unicode=true,
            pdftitle={Conclusion},
            pdfauthor={David Bowden},
            pdfborder={0 0 0},
            breaklinks=true}
\urlstyle{same}  % don't use monospace font for urls
\usepackage{longtable,booktabs}
\usepackage{graphicx,grffile}
\makeatletter
\def\maxwidth{\ifdim\Gin@nat@width>\linewidth\linewidth\else\Gin@nat@width\fi}
\def\maxheight{\ifdim\Gin@nat@height>\textheight\textheight\else\Gin@nat@height\fi}
\makeatother
% Scale images if necessary, so that they will not overflow the page
% margins by default, and it is still possible to overwrite the defaults
% using explicit options in \includegraphics[width, height, ...]{}
\setkeys{Gin}{width=\maxwidth,height=\maxheight,keepaspectratio}
\IfFileExists{parskip.sty}{%
\usepackage{parskip}
}{% else
\setlength{\parindent}{0pt}
\setlength{\parskip}{6pt plus 2pt minus 1pt}
}
\setlength{\emergencystretch}{3em}  % prevent overfull lines
\providecommand{\tightlist}{%
  \setlength{\itemsep}{0pt}\setlength{\parskip}{0pt}}
\setcounter{secnumdepth}{5}
% Redefines (sub)paragraphs to behave more like sections
\ifx\paragraph\undefined\else
\let\oldparagraph\paragraph
\renewcommand{\paragraph}[1]{\oldparagraph{#1}\mbox{}}
\fi
\ifx\subparagraph\undefined\else
\let\oldsubparagraph\subparagraph
\renewcommand{\subparagraph}[1]{\oldsubparagraph{#1}\mbox{}}
\fi

%%% Use protect on footnotes to avoid problems with footnotes in titles
\let\rmarkdownfootnote\footnote%
\def\footnote{\protect\rmarkdownfootnote}

%%% Change title format to be more compact
\usepackage{titling}

% Create subtitle command for use in maketitle
\newcommand{\subtitle}[1]{
  \posttitle{
    \begin{center}\large#1\end{center}
    }
}

\setlength{\droptitle}{-2em}
  \title{Conclusion}
  \pretitle{\vspace{\droptitle}\centering\huge}
  \posttitle{\par}
  \author{David Bowden}
  \preauthor{\centering\large\emph}
  \postauthor{\par}
  \predate{\centering\large\emph}
  \postdate{\par}
  \date{July 1, 2017}

\usepackage{setspace}

\usepackage{float}
\let\origtable\table
\let\endorigtable\endtable
\renewenvironment{table}[1][2] {
    \singlespacing
    \expandafter\origtable\expandafter[H]
} {
    \endorigtable
}

\frontmatter

\usepackage{amsthm}
\newtheorem{theorem}{Theorem}[chapter]
\newtheorem{lemma}{Lemma}[chapter]
\theoremstyle{definition}
\newtheorem{definition}{Definition}[chapter]
\newtheorem{corollary}{Corollary}[chapter]
\newtheorem{proposition}{Proposition}[chapter]
\theoremstyle{definition}
\newtheorem{example}{Example}[chapter]
\theoremstyle{remark}
\newtheorem*{remark}{Remark}
\begin{document}
\maketitle

\doublespacing

\mainmatter

\chapter{Conclusion}\label{conclusion}

Why do some conflicts have multiple rebel groups, while in other cases
dissidents form a single, cohesive group? As I discuss in Chapter
\ref{intro}, the importance of this question has been well established.
Civil wars with multiple rebel groups last longer than others
(Cunningham \protect\hyperlink{ref-Cunningham2006}{2006}; Akcinaroglu
\protect\hyperlink{ref-Akcinaroglu2012}{2012}), are less likely to end
in a peace agreement (Cunningham, Gleditsch, and Salehyan
\protect\hyperlink{ref-Cunningham2009}{2009}), have more bases on which
conflict could recur (Atlas and Licklider
\protect\hyperlink{ref-Atlas1999}{1999}), and produce more fatalities.
In short, civil wars with multiple rebel groups tend to be among the
most severe conflicts. Yet we know little about the causes of such
structures. No existing work addresses the formation of new rebel groups
during conflicts, and existing work on the splintering and merging of
existing rebel groups produces somewhat contradictory findings (see for
example Christia's (\protect\hyperlink{ref-Christia2012}{2012}) focus on
power versus Staniland's (\protect\hyperlink{ref-Staniland2014}{2014})
emphasis on social structure). My dissertation seeks to fill this gap in
the literature.

In Chapter \ref{theory} I articulate a theoretical framework of rebel
movement politics from which I derive predictions about rebel movement
structure. I start from the assumption that rebel groups are drawn from
a broader pool of dissidents, which includes peaceful activists in
addition to combatants. The loyalty of this dissident pool should be
crucially important to most rebel groups as a source of material
support, recruits, and political leverage. Rebel groups thus have an
incentive to be responsive to these individuals. Failure to represent
the interests of these non-violent dissidents will leave a rebel group
vulnerable to competition. New recruits may look to form a new rebel
group rather than joining an existing one, and entrepreneurial members
of existing rebel groups may form splinter organizations in hopes of
capturing the supporters of their previous organization. Thus it is the
interaction of the preferences of ordinary dissidents and the decisions
of rebel elites that determines rebel movement structure.

One circumstance in which rebel elites may fail to adequately adapt to
constituent preferences is the onset of repression. The threat of
physical violence should increase the risk of being a non-violent
dissident, and in turn decrease the \emph{relative} risk of fighting.
This should lead some individuals who previously declined to participate
in rebellion to take up arms. This influx of new recruits will not
always be a boon to existing rebel groups, however. Repression should
also tend to induce greater levels of ethnic identification, as
repression is often targeted disproportionately at certain ethnic
groups, ethnic groups often have militias and political organizations
that make them a useful basis for organizing defense against repression,
and appeals to co-ethnic states is often an effective means of securing
external support. Thus existing rebel groups may struggle to win over
these new recruits or even maintain their existing support, unless they
happen to already place strong emphasis on ethnic identity. Otherwise,
new organizations making more credible ethnicity-based appeals are
likely to attract the new recruits and steal civilian support from
existing rebel groups. Repression should therefore be associated with
both the formation of entirely new rebel groups, and of organizations
that splinter from existing rebel groups. To offset the loss of
capability that results from splintering, rebels should be open to
alliances and mergers with co-ethnic groups. In short, repression should
lead rebel movements to both grow and reorganize around ethnic identity.

I test the micro-level foundations of this theory in Chapter
\ref{survey-chapter} using data from the Afrobarometer survey.
Consistent with my expectations, I find that individuals who have
experienced an attack are more likely than others to express willingness
to participate in violence, and also more likely to identify with their
ethnic group rather than their nation. Greater levels of repression at
the national level are also associated with higher probabilities of
ethnic identification. The results hold after performing coarsened exact
matching, suggesting that there are not systematic observable
differences between individuals who have been attacked and individuals
who have not.

In Chapter \ref{entry} I examine the formation of new rebel groups
during ongoing conflicts. As I predict, the probability that new groups
will enter a conflict increases in response to increases in repression.
Contrary to my expectations, the ethnic diversity of a country does not
limit the scope of my theory --- new rebel groups form even at
relatively high and low levels of ethnic diversity. Adding support for
my theory is the finding that the rebel groups which join ongoing
conflicts are more likely than others to draw their support from a
single ethnic group. This suggests that the link between repression and
the formation of new groups is in fact related to ethnic identity,
rather than some alternative process. I supplement these quantitative
findings with a qualitative case study of the separatist movements in
Burma. The initiation of the separatist movement in Shan State strongly
supports my theory, as the rebellion emerged after a wave of abuses by
government forces, and placed a strong emphasis on Shan identity. The
Arakan case suggests several nuances, most notably the ability of
religion to create divisions within ethnic groups.

I test my predictions regarding splintering and alliance formation among
existing rebel groups in Chapter \ref{realignment}. Consistent with my
hypotheses, I find that increases in repression are associated with an
increased risk of splintering for existing rebel groups, though the
relationship is not completely robust. I do not find evidence for my
prediction that splinter organizations should be more likely than others
to draw their support from a single ethnic group. I find support for my
prediction that repression should increase the probability of
ethnically-homogeneous alliances forming, while it does not have the
hypothesized negative relationship with the formation of multi-ethnic
alliances. Burma again provides qualitative evidence in support of my
theory, as the emergence of splinter organizations such as the Karenni
National Progressive Party and Shan State Independence Army each appear
to be driven by a desire to provide stronger representation for their
respective ethnic groups. The formation of alliances among the Shan
rebels in response to a counterinsurgency campaign provides further
support for my framework.

\section{Implications}\label{implications}

The central implication of this research is that repression can trigger
a sectarian spiral, whereby previously non-violent individuals join the
fighting, and existing rebels reorganize around ethnic identity. I find
that repression increases the number of new rebel groups, splinter
organizations, and ethnically-homogeneous alliances. Given the rarity of
the latter, it is safe to assume that in most conflicts, repression
increases the total number of rebel groups.\footnote{A logit model (not
  reported) predicting which conflict years have multiple rebel groups
  without distinguishing between joiners, splinters, and alliances
  confirms this, as the level of repression is a strong predictor of
  multiple groups.} The level of repression against civilians explains a
substantial portion of the variation in the number of rebel groups in a
conflict, and I find multiple forms of evidence suggesting that the
mechanism is related to increased ethnic identification.

This conclusion contrasts with some prominent existing works. Christia
(\protect\hyperlink{ref-Christia2012}{2012}) argues that rebel
realignments are a function of the distribution of power between rebel
coalitions and the government. When rebels are weaker than the
government they will seek alliances. When rebels are stronger,
coalitions tend to fragment so as to minimize the members of people with
whom they must share private benefits. Her theory does not predict that
splintering and certain types of alliance formation would be closely
related, as I find them to be. This also contrasts with Kalyvas and
Kocher (\protect\hyperlink{ref-Kalyvas2007}{2007}), who expects that
rebel movements will generally become more cohesive over time. I show
that the trend is contingent on repression. While Christia
(\protect\hyperlink{ref-Christia2012}{2012}) does expect that ethnicity
should form an important component of the identity of new alliances, she
believes such identities are deployed instrumentally. My
individual-level findings suggest that the members and supporters of
rebel groups may sincerely adopt such identities, however, suggesting
that rebel elites cannot switch identities at will as Christia expects.
My findings are consistent with the work of Lewis
(\protect\hyperlink{ref-Lewis2016}{2016}), who argues that ethnicity is
not important to the initial organization of rebellion, but ethnic
rebellions are disproportionately likely to thrive. The findings here
suggest that rebellions without a clear ethnic identity should be
vulnerable to splintering and losing recruits to new, more explicitly
ethnic rival organizations.

My findings also contrast those of Staniland
(\protect\hyperlink{ref-Staniland2014}{2014}), who views internal social
structure as the key determinant of rebel group cohesion. I find instead
that an external factor, government repression, plays a surprisingly
large role in shaping rebel movement structure. To some extent, however,
this is a disagreement over the relative importance of the two factors.
Staniland (\protect\hyperlink{ref-Staniland2014}{2014}) essentially
assumes that repression will occur, and seeks to explain variation in
resilience to it. Still, I find that repression is generally a strong
predictor of splintering, while organizational characteristics are not.

This research also suggests a strong connection between the preferences
of rank-and-file dissidents and the broader patterns of rebel
organization. Existing work tends to conceptualize rebel groups as the
private armies of warlords (Christia
\protect\hyperlink{ref-Christia2012}{2012}), who maintain control either
through personal loyalty or the provision of private benefits (Lichbach
\protect\hyperlink{ref-Lichbach1995}{1995}; Weinstein
\protect\hyperlink{ref-Weinstein2007}{2007}). This viewpoint suggests
that rebel elites have little need to be responsive to their members. My
findings that both individual preferences and rebel movement structure
respond to repression suggests that ordinary rebels do in fact have a
consequential amount of agency. When leaders fail to accommodate their
preferences, rebel group members have exit options in the form of
splinter organizations and entirely new rebel groups. This implies that
there should be a surprising amount of accountability within rebel
organizations. At the same time, the formation of new rebel groups is
not uncommon, suggesting that rebel elites often fail to respond to
their members.

Finally, this research suggests at least one clear policy
recommendation. Previous work has shown that conflicts with multiple
rebel groups are especially severe and difficult to resolve. I find that
repression is a strong predictor of two different processes that can
result in greater numbers of rebel groups. This suggests that
governments facing rebellion might be well-served to refrain from
widespread repression, and instead target their counterinsurgency
operations against individuals who have already joined a rebel group to
the greatest extent possible. In the following section I discuss avenues
for future research on why governments repress despite these the
negative consequences. My results also bolster the argument that outside
states and the broader international community should endeavor to
protect civilians during civil wars. While doing so has long been
understood to be valuable from a humanitarian standpoint, my work
suggests that the potential for such policies to limit conflict severity
should place them in the self-interest neighboring states and any others
likely to be affected by the fighting. It should be noted, however, that
interventions of this sort are not foolproof. Notably, a UN effort to
create a humanitarian zone in Srebrenica, Bosnia in 1995 actually
facilitated the massacre of the civilians gathered there. These sorts of
humanitarian efforts should thus only be undertaken with a sufficiently
large deployment to ensure the security of the civilians under
protection.

\section{Future Research}\label{future-research}

While this project makes significant progress toward explaining rebel
movement structure, numerous avenues for future research remain. These
include both refinements to the analyses I present here, as well as new
analyses suggested by my results.

The individual-level analysis could be refined on several dimensions in
future work. One limitation of the existing results is their inability
to identify the source of repression. It would be possible to make
inferences about the likely perpetrator by matching the survey results,
which include the respondent's city, to a geocoded dataset of battles,
such as ACLED (Raleigh \protect\hyperlink{ref-Raleigh2012a}{2012}). If
most of the violent events in a particular locale are perpetrated by the
government, it might be reasonable to assume that it is the source of
most attacks on individuals in that area. By contrast, this would not be
a safe assumption in territory that is clearly controlled by a rebel
group. The use of an external conflict data source could also address
the issue of temporal ordering. The Afrobarometer data does not specify
whether individuals were attacked before or after they engaged in
violence themselves. With geocoded conflict data one could examine
whether the average probability of participation in violence or of
ethnic identification in a geographic area changes after violent events
there. Finally, a more robust method of causal inference that can
account for unobservable sources of bias would enhance the validity of
the results. While finding a valid instrument at the individual level
may be difficult, it should be possible to instrument for the
country-level human rights measure.

While general surveys such as the Afrobarometer provide useful data on
individual attitudes toward violence and ethnicity, they do not provide
tests of every element of my theory. Original survey or experimental
work exploring individual attitudes towards rebel groups would
potentially strengthen my arguments regarding the connection between
individual attitudes and rebel movement structure. For instance, a
finding that individuals who experience repression from the government
become less supportive of existing rebel groups would provide strong
support for my claim that dissident civilians are key drivers of change
to the configuration of the rebel movement.

The analysis of new rebel group formation could also benefit from
several improvements. Adding a causal inference technique to the
analysis would greatly enhance the validity of the results. While I did
not find oil revenue to be a viable instrumental variable, it is
possible that a suitable proxy for repression exists, such as colonial
history. An alternative option could be panel data techniques that
facilitate causal inference without the need for exogenous instruments
(Kim and Frees \protect\hyperlink{ref-Kim2007}{2007}). A more detailed
analysis of the attributes of the rebel groups that join ongoing
conflicts could also lend further support to my theoretical framework.
While the finding that joining groups are more likely than others to
draw support from a single ethnic group lends credibility to my
argument, an examination of the platform and recruiting appeals of these
groups could strengthen the argument that group formation is motivated
by a desire to place greater emphasis on ethnic identity. Relatedly, the
relationships between newly formed rebel groups and others should be
explored. Enhanced ethnic identification might lead to conflict between
rebel groups of differing ethnicities. Alternatively, competition for
civilian support might produce conflict between co-ethnic rebel groups.

The group formation chapter also raises important questions about
government strategy. My finding that repression tends to increase the
number of rebel groups makes its use by governments a puzzle. Future
work should examine the government's strategic calculus in more detail.
It is possible that repression has some hidden benefit that outweighs
the cost of additional rebel groups. My findings in Chapter
\ref{survey-chapter} that repression is negatively related to voting
suggests one possible answers --- governments are essentially accepting
an increase in the level of violence by dissidents in exchange for a
reduction in the overall size of the dissident movement. Relatedly,
while repression increases the number of individuals willing to use
violence, it also provokes division among dissidents along ethnic lines.
The latter consequence might be sufficiently desirable as part of a
divide-and-conquer strategy to justify the former. Finally, the
repression puzzle may be a result of incomplete information. It could be
the case the repression offers some possibility of total defeat of the
dissident movement, and governments accept the risk of inspiring new
rebel groups in pursuit of this outcome. In the Arab Spring, for
example, Bahrain used repression to quickly put down the opposition
movement there. While the tactic backfired in Syria, the possibility of
an outcome similar to Bahrain may have made it a worthwhile gamble.

The analysis of rebel group realignment also has room for improvement.
While the findings for both splintering and alliance formation are
mostly consistent with my predictions, the results are less robust than
would be ideal. This is likely due in part to the rarity of both
outcomes. This could likely be remedied, however, as the current
analysis only looks at the most extreme instances of splintering and
merging --- those which result in the formation of new rebel
organizations with distinct names. A less extreme, and likely more
prevalent form of splintering is the loss of membership, either to rival
rebel groups, or to desertion. While this phenomenon would be quite
difficult to measure for the entire post-World War II sample, it may be
possible to track changes in rebel group membership for a smaller sample
of conflicts. With respect to alliances, I consider only cases where
formerly independent rebel groups merge to a significant degree. There
are undoubtedly many instances of meaningful cooperation between rebel
groups that fall short of formal integration. Indeed, a forthcoming data
project (Asal and Rethemeyer \protect\hyperlink{ref-Asal2015}{2015})
should facilitate analysis of such behavior. Including these less
extreme examples of splintering and alliance formation should mitigate
concerns about the rarity of these outcomes. The concerns from the group
formation chapter also apply, as this analysis would benefit from a
causal inference strategy and closer inspection of the rationale that
rebel elites use to justify the creation of their new groups.

At a broader level, other processes affecting rebel movement structure
should be explored. I do not claim to provide a complete account of
rebel movement structure, but rather a probabilistic theory of what I
believe to be one of the most common pathways to multiple rebel groups.
Other factors undoubtedly operate in some cases. For example, many
instances of splintering occur not over ethnic lines, but over a divide
between moderates who wish to participate in a peace process, and
hardliners who wish to continue fighting. While this phenomenon has
received some attention in the context of negotiating peace agreements
(Stedman \protect\hyperlink{ref-Stedman1997}{1997}), there is little
work on the conditions under which this is likely, nor on the
implications for the subsequent fighting. I argue that attracting
external support may be one reason for increased ethnic identification
following repression, but do not explore external sponsorship in detail.
In some cases external states may have a substantial amount of agency,
however, which merits greater attention. For example, the Gulf
Cooperation Council has repeatedly sought to establish an alliance of
relatively moderate Sunni rebel groups in Syria. Finally, future work
should explore factors that run in the opposite direction, fostering
greater cohesion within rebel movements. The Latin American rebellions
are generally much more cohesive than those in other regions. The
explanation could relate to my theory, perhaps being the result of more
fluid ethnic identities than are seen in most parts of the world.
Alternatively, the explanation might involve the extreme levels of
external sponsorship seen during the Cold War, or some
as-yet-undiscovered factor.

Finally, the severity of conflicts that experience this cycle of
increased ethnic identification suggests a need for research on ways to
reverse the process. Increased sectarianism can increase conflict
severity, and as we have seen in places such as Afghanistan and the
Democratic Republic of the Congo, can hinder the prospects for lasting
peace. Preventing repression is an obvious policy recommendation of this
research. Yet, that is more easily said than done, and is of little use
in cases where it has already occurred. The most commonly cited factor
that can increase national unity is external conflict (Tilly
\protect\hyperlink{ref-Tilly1992}{1992}; Gibler, Hutchison, and Miller
\protect\hyperlink{ref-Gibler2012}{2012}). Obviously, however, this is
not a tenable solution. A few studies have suggested that economic
development might promote national identities at the expense of ethnic
ones (Miguel \protect\hyperlink{ref-Miguel2004b}{2004}), but much more
research is needed in this area.

\chapter*{References}\label{references}
\addcontentsline{toc}{chapter}{References}

\markboth{REFERENCES}{}

\indent

\setlength{\parindent}{-0.2in} \setlength{\leftskip}{0.2in}
\setlength{\parskip}{8pt}

\singlespacing

\hypertarget{refs}{}
\hypertarget{ref-Akcinaroglu2012}{}
Akcinaroglu, Seden. 2012. ``Rebel Interdependencies and Civil War
Outcomes.'' \emph{Journal of Conflict Resolution} 56 (5): 879--903.
doi:\href{https://doi.org/10.1177/0022002712445741}{10.1177/0022002712445741}.

\hypertarget{ref-Asal2015}{}
Asal, Victor H., and R. Karl Rethemeyer. 2015. ``Big Allied and
Dangerous Dataset, version 2.''

\hypertarget{ref-Atlas1999}{}
Atlas, Pierre M., and Roy Licklider. 1999. ``Conflict Among Former
Allies After Civil War Settlement : Sudan , Zimbabwe , Chad , and
Lebanon.'' \emph{Journal of Peace Research} 36 (1): 35--54.

\hypertarget{ref-Christia2012}{}
Christia, Fotini. 2012. \emph{Alliance Formation in Civil Wars}.
Cambridge: Cambridge University Press.

\hypertarget{ref-Cunningham2006}{}
Cunningham, David E. 2006. ``Veto Players and Civil War Duration.''
\emph{American Journal of Political Science} 50 (4): 875--92.

\hypertarget{ref-Cunningham2009}{}
Cunningham, David E., Kristian Skrede Gleditsch, and Idean Salehyan.
2009. ``It Takes Two: A Dyadic Analysis of Civil War Duration and
Outcome.'' \emph{Journal of Conflict Resolution} 53 (4): 570--97.

\hypertarget{ref-Gibler2012}{}
Gibler, Douglas M., Marc L. Hutchison, and Steven V. Miller. 2012.
``Individual Identity Attachments and International Conflict: The
Importance of Territorial Threat.'' \emph{Comparative Political Studies}
45 (12): 1655--83.

\hypertarget{ref-Kalyvas2007}{}
Kalyvas, Stathis N., and Matthew Adam Kocher. 2007. ``How `Free' Is Free
Riding in Civil Wars? Violence, Insurgency, and the Collective Action
Problem.'' \emph{World Politics} 59 (2): 177--216.

\hypertarget{ref-Kim2007}{}
Kim, Jee Seon, and Edward W. Frees. 2007. ``Multilevel modeling with
correlated effects.'' \emph{Psychometrika} 72 (4): 505--33.

\hypertarget{ref-Lewis2016}{}
Lewis, Janet I. 2016. ``How Does Ethnic Rebellion Start ?''
\emph{Comparative Political Studies}, forthcoming.

\hypertarget{ref-Lichbach1995}{}
Lichbach, Mark Irving. 1995. \emph{The Rebel's Dilemma}. Ann Arbor, MI:
University of Michigan Press.

\hypertarget{ref-Miguel2004b}{}
Miguel, Edward. 2004. ``Tribe or Nation? Nation Building and Public
Goods in Kenya versus Tanzania.'' \emph{World Politics} 56 (3): 327--62.
doi:\href{https://doi.org/10.1353/wp.2004.0018}{10.1353/wp.2004.0018}.

\hypertarget{ref-Raleigh2012a}{}
Raleigh, Clionadh. 2012. ``Violence Against Civilians: A Disaggregated
Analysis.'' \emph{International Interactions} 38 (4): 462--81.

\hypertarget{ref-Staniland2014}{}
Staniland, Paul. 2014. \emph{Networks of Rebellion: Explaining Insurgent
Cohesion and Collapse}. Ithaca, NY: Cornell University Press.

\hypertarget{ref-Stedman1997}{}
Stedman, Stephen John. 1997. ``Spoiler Problems in Peace Processes.''
\emph{International Security} 22 (2): 5.

\hypertarget{ref-Tilly1992}{}
Tilly, Charles. 1992. \emph{Coercion, capital, and European states, AD
990-1992}. Malden, MA: Blackwell.

\hypertarget{ref-Weinstein2007}{}
Weinstein, Jeremy M. 2007. \emph{Inside Rebellion}. Cambridge: Cambridge
University Press.


\end{document}
